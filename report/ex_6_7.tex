\documentclass[11pt]{article}

    \usepackage[breakable]{tcolorbox}
    \usepackage{parskip} % Stop auto-indenting (to mimic markdown behaviour)
    
    \usepackage{iftex}
    \ifPDFTeX
    	\usepackage[T1]{fontenc}
    	\usepackage{mathpazo}
    \else
    	\usepackage{fontspec}
    \fi

    % Basic figure setup, for now with no caption control since it's done
    % automatically by Pandoc (which extracts ![](path) syntax from Markdown).
    \usepackage{graphicx}
    % Maintain compatibility with old templates. Remove in nbconvert 6.0
    \let\Oldincludegraphics\includegraphics
    % Ensure that by default, figures have no caption (until we provide a
    % proper Figure object with a Caption API and a way to capture that
    % in the conversion process - todo).
    \usepackage{caption}
    \DeclareCaptionFormat{nocaption}{}
    \captionsetup{format=nocaption,aboveskip=0pt,belowskip=0pt}

    \usepackage{float}
    \floatplacement{figure}{H} % forces figures to be placed at the correct location
    \usepackage{xcolor} % Allow colors to be defined
    \usepackage{enumerate} % Needed for markdown enumerations to work
    \usepackage{geometry} % Used to adjust the document margins
    \usepackage{amsmath} % Equations
    \usepackage{amssymb} % Equations
    \usepackage{textcomp} % defines textquotesingle
    % Hack from http://tex.stackexchange.com/a/47451/13684:
    \AtBeginDocument{%
        \def\PYZsq{\textquotesingle}% Upright quotes in Pygmentized code
    }
    \usepackage{upquote} % Upright quotes for verbatim code
    \usepackage{eurosym} % defines \euro
    \usepackage[mathletters]{ucs} % Extended unicode (utf-8) support
    \usepackage{fancyvrb} % verbatim replacement that allows latex
    \usepackage{grffile} % extends the file name processing of package graphics 
                         % to support a larger range
    \makeatletter % fix for old versions of grffile with XeLaTeX
    \@ifpackagelater{grffile}{2019/11/01}
    {
      % Do nothing on new versions
    }
    {
      \def\Gread@@xetex#1{%
        \IfFileExists{"\Gin@base".bb}%
        {\Gread@eps{\Gin@base.bb}}%
        {\Gread@@xetex@aux#1}%
      }
    }
    \makeatother
    \usepackage[Export]{adjustbox} % Used to constrain images to a maximum size
    \adjustboxset{max size={0.9\linewidth}{0.9\paperheight}}

    % The hyperref package gives us a pdf with properly built
    % internal navigation ('pdf bookmarks' for the table of contents,
    % internal cross-reference links, web links for URLs, etc.)
    \usepackage{hyperref}
    % The default LaTeX title has an obnoxious amount of whitespace. By default,
    % titling removes some of it. It also provides customization options.
    \usepackage{titling}
    \usepackage{longtable} % longtable support required by pandoc >1.10
    \usepackage{booktabs}  % table support for pandoc > 1.12.2
    \usepackage[inline]{enumitem} % IRkernel/repr support (it uses the enumerate* environment)
    \usepackage[normalem]{ulem} % ulem is needed to support strikethroughs (\sout)
                                % normalem makes italics be italics, not underlines
    \usepackage{mathrsfs}
    

    
    % Colors for the hyperref package
    \definecolor{urlcolor}{rgb}{0,.145,.698}
    \definecolor{linkcolor}{rgb}{.71,0.21,0.01}
    \definecolor{citecolor}{rgb}{.12,.54,.11}

    % ANSI colors
    \definecolor{ansi-black}{HTML}{3E424D}
    \definecolor{ansi-black-intense}{HTML}{282C36}
    \definecolor{ansi-red}{HTML}{E75C58}
    \definecolor{ansi-red-intense}{HTML}{B22B31}
    \definecolor{ansi-green}{HTML}{00A250}
    \definecolor{ansi-green-intense}{HTML}{007427}
    \definecolor{ansi-yellow}{HTML}{DDB62B}
    \definecolor{ansi-yellow-intense}{HTML}{B27D12}
    \definecolor{ansi-blue}{HTML}{208FFB}
    \definecolor{ansi-blue-intense}{HTML}{0065CA}
    \definecolor{ansi-magenta}{HTML}{D160C4}
    \definecolor{ansi-magenta-intense}{HTML}{A03196}
    \definecolor{ansi-cyan}{HTML}{60C6C8}
    \definecolor{ansi-cyan-intense}{HTML}{258F8F}
    \definecolor{ansi-white}{HTML}{C5C1B4}
    \definecolor{ansi-white-intense}{HTML}{A1A6B2}
    \definecolor{ansi-default-inverse-fg}{HTML}{FFFFFF}
    \definecolor{ansi-default-inverse-bg}{HTML}{000000}

    % common color for the border for error outputs.
    \definecolor{outerrorbackground}{HTML}{FFDFDF}

    % commands and environments needed by pandoc snippets
    % extracted from the output of `pandoc -s`
    \providecommand{\tightlist}{%
      \setlength{\itemsep}{0pt}\setlength{\parskip}{0pt}}
    \DefineVerbatimEnvironment{Highlighting}{Verbatim}{commandchars=\\\{\}}
    % Add ',fontsize=\small' for more characters per line
    \newenvironment{Shaded}{}{}
    \newcommand{\KeywordTok}[1]{\textcolor[rgb]{0.00,0.44,0.13}{\textbf{{#1}}}}
    \newcommand{\DataTypeTok}[1]{\textcolor[rgb]{0.56,0.13,0.00}{{#1}}}
    \newcommand{\DecValTok}[1]{\textcolor[rgb]{0.25,0.63,0.44}{{#1}}}
    \newcommand{\BaseNTok}[1]{\textcolor[rgb]{0.25,0.63,0.44}{{#1}}}
    \newcommand{\FloatTok}[1]{\textcolor[rgb]{0.25,0.63,0.44}{{#1}}}
    \newcommand{\CharTok}[1]{\textcolor[rgb]{0.25,0.44,0.63}{{#1}}}
    \newcommand{\StringTok}[1]{\textcolor[rgb]{0.25,0.44,0.63}{{#1}}}
    \newcommand{\CommentTok}[1]{\textcolor[rgb]{0.38,0.63,0.69}{\textit{{#1}}}}
    \newcommand{\OtherTok}[1]{\textcolor[rgb]{0.00,0.44,0.13}{{#1}}}
    \newcommand{\AlertTok}[1]{\textcolor[rgb]{1.00,0.00,0.00}{\textbf{{#1}}}}
    \newcommand{\FunctionTok}[1]{\textcolor[rgb]{0.02,0.16,0.49}{{#1}}}
    \newcommand{\RegionMarkerTok}[1]{{#1}}
    \newcommand{\ErrorTok}[1]{\textcolor[rgb]{1.00,0.00,0.00}{\textbf{{#1}}}}
    \newcommand{\NormalTok}[1]{{#1}}
    
    % Additional commands for more recent versions of Pandoc
    \newcommand{\ConstantTok}[1]{\textcolor[rgb]{0.53,0.00,0.00}{{#1}}}
    \newcommand{\SpecialCharTok}[1]{\textcolor[rgb]{0.25,0.44,0.63}{{#1}}}
    \newcommand{\VerbatimStringTok}[1]{\textcolor[rgb]{0.25,0.44,0.63}{{#1}}}
    \newcommand{\SpecialStringTok}[1]{\textcolor[rgb]{0.73,0.40,0.53}{{#1}}}
    \newcommand{\ImportTok}[1]{{#1}}
    \newcommand{\DocumentationTok}[1]{\textcolor[rgb]{0.73,0.13,0.13}{\textit{{#1}}}}
    \newcommand{\AnnotationTok}[1]{\textcolor[rgb]{0.38,0.63,0.69}{\textbf{\textit{{#1}}}}}
    \newcommand{\CommentVarTok}[1]{\textcolor[rgb]{0.38,0.63,0.69}{\textbf{\textit{{#1}}}}}
    \newcommand{\VariableTok}[1]{\textcolor[rgb]{0.10,0.09,0.49}{{#1}}}
    \newcommand{\ControlFlowTok}[1]{\textcolor[rgb]{0.00,0.44,0.13}{\textbf{{#1}}}}
    \newcommand{\OperatorTok}[1]{\textcolor[rgb]{0.40,0.40,0.40}{{#1}}}
    \newcommand{\BuiltInTok}[1]{{#1}}
    \newcommand{\ExtensionTok}[1]{{#1}}
    \newcommand{\PreprocessorTok}[1]{\textcolor[rgb]{0.74,0.48,0.00}{{#1}}}
    \newcommand{\AttributeTok}[1]{\textcolor[rgb]{0.49,0.56,0.16}{{#1}}}
    \newcommand{\InformationTok}[1]{\textcolor[rgb]{0.38,0.63,0.69}{\textbf{\textit{{#1}}}}}
    \newcommand{\WarningTok}[1]{\textcolor[rgb]{0.38,0.63,0.69}{\textbf{\textit{{#1}}}}}
    
    
    % Define a nice break command that doesn't care if a line doesn't already
    % exist.
    \def\br{\hspace*{\fill} \\* }
    % Math Jax compatibility definitions
    \def\gt{>}
    \def\lt{<}
    \let\Oldtex\TeX
    \let\Oldlatex\LaTeX
    \renewcommand{\TeX}{\textrm{\Oldtex}}
    \renewcommand{\LaTeX}{\textrm{\Oldlatex}}
    % Document parameters
    % Document title
    \title{ex\_6\_7}
    
    
    
    
    
% Pygments definitions
\makeatletter
\def\PY@reset{\let\PY@it=\relax \let\PY@bf=\relax%
    \let\PY@ul=\relax \let\PY@tc=\relax%
    \let\PY@bc=\relax \let\PY@ff=\relax}
\def\PY@tok#1{\csname PY@tok@#1\endcsname}
\def\PY@toks#1+{\ifx\relax#1\empty\else%
    \PY@tok{#1}\expandafter\PY@toks\fi}
\def\PY@do#1{\PY@bc{\PY@tc{\PY@ul{%
    \PY@it{\PY@bf{\PY@ff{#1}}}}}}}
\def\PY#1#2{\PY@reset\PY@toks#1+\relax+\PY@do{#2}}

\@namedef{PY@tok@w}{\def\PY@tc##1{\textcolor[rgb]{0.73,0.73,0.73}{##1}}}
\@namedef{PY@tok@c}{\let\PY@it=\textit\def\PY@tc##1{\textcolor[rgb]{0.25,0.50,0.50}{##1}}}
\@namedef{PY@tok@cp}{\def\PY@tc##1{\textcolor[rgb]{0.74,0.48,0.00}{##1}}}
\@namedef{PY@tok@k}{\let\PY@bf=\textbf\def\PY@tc##1{\textcolor[rgb]{0.00,0.50,0.00}{##1}}}
\@namedef{PY@tok@kp}{\def\PY@tc##1{\textcolor[rgb]{0.00,0.50,0.00}{##1}}}
\@namedef{PY@tok@kt}{\def\PY@tc##1{\textcolor[rgb]{0.69,0.00,0.25}{##1}}}
\@namedef{PY@tok@o}{\def\PY@tc##1{\textcolor[rgb]{0.40,0.40,0.40}{##1}}}
\@namedef{PY@tok@ow}{\let\PY@bf=\textbf\def\PY@tc##1{\textcolor[rgb]{0.67,0.13,1.00}{##1}}}
\@namedef{PY@tok@nb}{\def\PY@tc##1{\textcolor[rgb]{0.00,0.50,0.00}{##1}}}
\@namedef{PY@tok@nf}{\def\PY@tc##1{\textcolor[rgb]{0.00,0.00,1.00}{##1}}}
\@namedef{PY@tok@nc}{\let\PY@bf=\textbf\def\PY@tc##1{\textcolor[rgb]{0.00,0.00,1.00}{##1}}}
\@namedef{PY@tok@nn}{\let\PY@bf=\textbf\def\PY@tc##1{\textcolor[rgb]{0.00,0.00,1.00}{##1}}}
\@namedef{PY@tok@ne}{\let\PY@bf=\textbf\def\PY@tc##1{\textcolor[rgb]{0.82,0.25,0.23}{##1}}}
\@namedef{PY@tok@nv}{\def\PY@tc##1{\textcolor[rgb]{0.10,0.09,0.49}{##1}}}
\@namedef{PY@tok@no}{\def\PY@tc##1{\textcolor[rgb]{0.53,0.00,0.00}{##1}}}
\@namedef{PY@tok@nl}{\def\PY@tc##1{\textcolor[rgb]{0.63,0.63,0.00}{##1}}}
\@namedef{PY@tok@ni}{\let\PY@bf=\textbf\def\PY@tc##1{\textcolor[rgb]{0.60,0.60,0.60}{##1}}}
\@namedef{PY@tok@na}{\def\PY@tc##1{\textcolor[rgb]{0.49,0.56,0.16}{##1}}}
\@namedef{PY@tok@nt}{\let\PY@bf=\textbf\def\PY@tc##1{\textcolor[rgb]{0.00,0.50,0.00}{##1}}}
\@namedef{PY@tok@nd}{\def\PY@tc##1{\textcolor[rgb]{0.67,0.13,1.00}{##1}}}
\@namedef{PY@tok@s}{\def\PY@tc##1{\textcolor[rgb]{0.73,0.13,0.13}{##1}}}
\@namedef{PY@tok@sd}{\let\PY@it=\textit\def\PY@tc##1{\textcolor[rgb]{0.73,0.13,0.13}{##1}}}
\@namedef{PY@tok@si}{\let\PY@bf=\textbf\def\PY@tc##1{\textcolor[rgb]{0.73,0.40,0.53}{##1}}}
\@namedef{PY@tok@se}{\let\PY@bf=\textbf\def\PY@tc##1{\textcolor[rgb]{0.73,0.40,0.13}{##1}}}
\@namedef{PY@tok@sr}{\def\PY@tc##1{\textcolor[rgb]{0.73,0.40,0.53}{##1}}}
\@namedef{PY@tok@ss}{\def\PY@tc##1{\textcolor[rgb]{0.10,0.09,0.49}{##1}}}
\@namedef{PY@tok@sx}{\def\PY@tc##1{\textcolor[rgb]{0.00,0.50,0.00}{##1}}}
\@namedef{PY@tok@m}{\def\PY@tc##1{\textcolor[rgb]{0.40,0.40,0.40}{##1}}}
\@namedef{PY@tok@gh}{\let\PY@bf=\textbf\def\PY@tc##1{\textcolor[rgb]{0.00,0.00,0.50}{##1}}}
\@namedef{PY@tok@gu}{\let\PY@bf=\textbf\def\PY@tc##1{\textcolor[rgb]{0.50,0.00,0.50}{##1}}}
\@namedef{PY@tok@gd}{\def\PY@tc##1{\textcolor[rgb]{0.63,0.00,0.00}{##1}}}
\@namedef{PY@tok@gi}{\def\PY@tc##1{\textcolor[rgb]{0.00,0.63,0.00}{##1}}}
\@namedef{PY@tok@gr}{\def\PY@tc##1{\textcolor[rgb]{1.00,0.00,0.00}{##1}}}
\@namedef{PY@tok@ge}{\let\PY@it=\textit}
\@namedef{PY@tok@gs}{\let\PY@bf=\textbf}
\@namedef{PY@tok@gp}{\let\PY@bf=\textbf\def\PY@tc##1{\textcolor[rgb]{0.00,0.00,0.50}{##1}}}
\@namedef{PY@tok@go}{\def\PY@tc##1{\textcolor[rgb]{0.53,0.53,0.53}{##1}}}
\@namedef{PY@tok@gt}{\def\PY@tc##1{\textcolor[rgb]{0.00,0.27,0.87}{##1}}}
\@namedef{PY@tok@err}{\def\PY@bc##1{{\setlength{\fboxsep}{-\fboxrule}\fcolorbox[rgb]{1.00,0.00,0.00}{1,1,1}{\strut ##1}}}}
\@namedef{PY@tok@kc}{\let\PY@bf=\textbf\def\PY@tc##1{\textcolor[rgb]{0.00,0.50,0.00}{##1}}}
\@namedef{PY@tok@kd}{\let\PY@bf=\textbf\def\PY@tc##1{\textcolor[rgb]{0.00,0.50,0.00}{##1}}}
\@namedef{PY@tok@kn}{\let\PY@bf=\textbf\def\PY@tc##1{\textcolor[rgb]{0.00,0.50,0.00}{##1}}}
\@namedef{PY@tok@kr}{\let\PY@bf=\textbf\def\PY@tc##1{\textcolor[rgb]{0.00,0.50,0.00}{##1}}}
\@namedef{PY@tok@bp}{\def\PY@tc##1{\textcolor[rgb]{0.00,0.50,0.00}{##1}}}
\@namedef{PY@tok@fm}{\def\PY@tc##1{\textcolor[rgb]{0.00,0.00,1.00}{##1}}}
\@namedef{PY@tok@vc}{\def\PY@tc##1{\textcolor[rgb]{0.10,0.09,0.49}{##1}}}
\@namedef{PY@tok@vg}{\def\PY@tc##1{\textcolor[rgb]{0.10,0.09,0.49}{##1}}}
\@namedef{PY@tok@vi}{\def\PY@tc##1{\textcolor[rgb]{0.10,0.09,0.49}{##1}}}
\@namedef{PY@tok@vm}{\def\PY@tc##1{\textcolor[rgb]{0.10,0.09,0.49}{##1}}}
\@namedef{PY@tok@sa}{\def\PY@tc##1{\textcolor[rgb]{0.73,0.13,0.13}{##1}}}
\@namedef{PY@tok@sb}{\def\PY@tc##1{\textcolor[rgb]{0.73,0.13,0.13}{##1}}}
\@namedef{PY@tok@sc}{\def\PY@tc##1{\textcolor[rgb]{0.73,0.13,0.13}{##1}}}
\@namedef{PY@tok@dl}{\def\PY@tc##1{\textcolor[rgb]{0.73,0.13,0.13}{##1}}}
\@namedef{PY@tok@s2}{\def\PY@tc##1{\textcolor[rgb]{0.73,0.13,0.13}{##1}}}
\@namedef{PY@tok@sh}{\def\PY@tc##1{\textcolor[rgb]{0.73,0.13,0.13}{##1}}}
\@namedef{PY@tok@s1}{\def\PY@tc##1{\textcolor[rgb]{0.73,0.13,0.13}{##1}}}
\@namedef{PY@tok@mb}{\def\PY@tc##1{\textcolor[rgb]{0.40,0.40,0.40}{##1}}}
\@namedef{PY@tok@mf}{\def\PY@tc##1{\textcolor[rgb]{0.40,0.40,0.40}{##1}}}
\@namedef{PY@tok@mh}{\def\PY@tc##1{\textcolor[rgb]{0.40,0.40,0.40}{##1}}}
\@namedef{PY@tok@mi}{\def\PY@tc##1{\textcolor[rgb]{0.40,0.40,0.40}{##1}}}
\@namedef{PY@tok@il}{\def\PY@tc##1{\textcolor[rgb]{0.40,0.40,0.40}{##1}}}
\@namedef{PY@tok@mo}{\def\PY@tc##1{\textcolor[rgb]{0.40,0.40,0.40}{##1}}}
\@namedef{PY@tok@ch}{\let\PY@it=\textit\def\PY@tc##1{\textcolor[rgb]{0.25,0.50,0.50}{##1}}}
\@namedef{PY@tok@cm}{\let\PY@it=\textit\def\PY@tc##1{\textcolor[rgb]{0.25,0.50,0.50}{##1}}}
\@namedef{PY@tok@cpf}{\let\PY@it=\textit\def\PY@tc##1{\textcolor[rgb]{0.25,0.50,0.50}{##1}}}
\@namedef{PY@tok@c1}{\let\PY@it=\textit\def\PY@tc##1{\textcolor[rgb]{0.25,0.50,0.50}{##1}}}
\@namedef{PY@tok@cs}{\let\PY@it=\textit\def\PY@tc##1{\textcolor[rgb]{0.25,0.50,0.50}{##1}}}

\def\PYZbs{\char`\\}
\def\PYZus{\char`\_}
\def\PYZob{\char`\{}
\def\PYZcb{\char`\}}
\def\PYZca{\char`\^}
\def\PYZam{\char`\&}
\def\PYZlt{\char`\<}
\def\PYZgt{\char`\>}
\def\PYZsh{\char`\#}
\def\PYZpc{\char`\%}
\def\PYZdl{\char`\$}
\def\PYZhy{\char`\-}
\def\PYZsq{\char`\'}
\def\PYZdq{\char`\"}
\def\PYZti{\char`\~}
% for compatibility with earlier versions
\def\PYZat{@}
\def\PYZlb{[}
\def\PYZrb{]}
\makeatother


    % For linebreaks inside Verbatim environment from package fancyvrb. 
    \makeatletter
        \newbox\Wrappedcontinuationbox 
        \newbox\Wrappedvisiblespacebox 
        \newcommand*\Wrappedvisiblespace {\textcolor{red}{\textvisiblespace}} 
        \newcommand*\Wrappedcontinuationsymbol {\textcolor{red}{\llap{\tiny$\m@th\hookrightarrow$}}} 
        \newcommand*\Wrappedcontinuationindent {3ex } 
        \newcommand*\Wrappedafterbreak {\kern\Wrappedcontinuationindent\copy\Wrappedcontinuationbox} 
        % Take advantage of the already applied Pygments mark-up to insert 
        % potential linebreaks for TeX processing. 
        %        {, <, #, %, $, ' and ": go to next line. 
        %        _, }, ^, &, >, - and ~: stay at end of broken line. 
        % Use of \textquotesingle for straight quote. 
        \newcommand*\Wrappedbreaksatspecials {% 
            \def\PYGZus{\discretionary{\char`\_}{\Wrappedafterbreak}{\char`\_}}% 
            \def\PYGZob{\discretionary{}{\Wrappedafterbreak\char`\{}{\char`\{}}% 
            \def\PYGZcb{\discretionary{\char`\}}{\Wrappedafterbreak}{\char`\}}}% 
            \def\PYGZca{\discretionary{\char`\^}{\Wrappedafterbreak}{\char`\^}}% 
            \def\PYGZam{\discretionary{\char`\&}{\Wrappedafterbreak}{\char`\&}}% 
            \def\PYGZlt{\discretionary{}{\Wrappedafterbreak\char`\<}{\char`\<}}% 
            \def\PYGZgt{\discretionary{\char`\>}{\Wrappedafterbreak}{\char`\>}}% 
            \def\PYGZsh{\discretionary{}{\Wrappedafterbreak\char`\#}{\char`\#}}% 
            \def\PYGZpc{\discretionary{}{\Wrappedafterbreak\char`\%}{\char`\%}}% 
            \def\PYGZdl{\discretionary{}{\Wrappedafterbreak\char`\$}{\char`\$}}% 
            \def\PYGZhy{\discretionary{\char`\-}{\Wrappedafterbreak}{\char`\-}}% 
            \def\PYGZsq{\discretionary{}{\Wrappedafterbreak\textquotesingle}{\textquotesingle}}% 
            \def\PYGZdq{\discretionary{}{\Wrappedafterbreak\char`\"}{\char`\"}}% 
            \def\PYGZti{\discretionary{\char`\~}{\Wrappedafterbreak}{\char`\~}}% 
        } 
        % Some characters . , ; ? ! / are not pygmentized. 
        % This macro makes them "active" and they will insert potential linebreaks 
        \newcommand*\Wrappedbreaksatpunct {% 
            \lccode`\~`\.\lowercase{\def~}{\discretionary{\hbox{\char`\.}}{\Wrappedafterbreak}{\hbox{\char`\.}}}% 
            \lccode`\~`\,\lowercase{\def~}{\discretionary{\hbox{\char`\,}}{\Wrappedafterbreak}{\hbox{\char`\,}}}% 
            \lccode`\~`\;\lowercase{\def~}{\discretionary{\hbox{\char`\;}}{\Wrappedafterbreak}{\hbox{\char`\;}}}% 
            \lccode`\~`\:\lowercase{\def~}{\discretionary{\hbox{\char`\:}}{\Wrappedafterbreak}{\hbox{\char`\:}}}% 
            \lccode`\~`\?\lowercase{\def~}{\discretionary{\hbox{\char`\?}}{\Wrappedafterbreak}{\hbox{\char`\?}}}% 
            \lccode`\~`\!\lowercase{\def~}{\discretionary{\hbox{\char`\!}}{\Wrappedafterbreak}{\hbox{\char`\!}}}% 
            \lccode`\~`\/\lowercase{\def~}{\discretionary{\hbox{\char`\/}}{\Wrappedafterbreak}{\hbox{\char`\/}}}% 
            \catcode`\.\active
            \catcode`\,\active 
            \catcode`\;\active
            \catcode`\:\active
            \catcode`\?\active
            \catcode`\!\active
            \catcode`\/\active 
            \lccode`\~`\~ 	
        }
    \makeatother

    \let\OriginalVerbatim=\Verbatim
    \makeatletter
    \renewcommand{\Verbatim}[1][1]{%
        %\parskip\z@skip
        \sbox\Wrappedcontinuationbox {\Wrappedcontinuationsymbol}%
        \sbox\Wrappedvisiblespacebox {\FV@SetupFont\Wrappedvisiblespace}%
        \def\FancyVerbFormatLine ##1{\hsize\linewidth
            \vtop{\raggedright\hyphenpenalty\z@\exhyphenpenalty\z@
                \doublehyphendemerits\z@\finalhyphendemerits\z@
                \strut ##1\strut}%
        }%
        % If the linebreak is at a space, the latter will be displayed as visible
        % space at end of first line, and a continuation symbol starts next line.
        % Stretch/shrink are however usually zero for typewriter font.
        \def\FV@Space {%
            \nobreak\hskip\z@ plus\fontdimen3\font minus\fontdimen4\font
            \discretionary{\copy\Wrappedvisiblespacebox}{\Wrappedafterbreak}
            {\kern\fontdimen2\font}%
        }%
        
        % Allow breaks at special characters using \PYG... macros.
        \Wrappedbreaksatspecials
        % Breaks at punctuation characters . , ; ? ! and / need catcode=\active 	
        \OriginalVerbatim[#1,codes*=\Wrappedbreaksatpunct]%
    }
    \makeatother

    % Exact colors from NB
    \definecolor{incolor}{HTML}{303F9F}
    \definecolor{outcolor}{HTML}{D84315}
    \definecolor{cellborder}{HTML}{CFCFCF}
    \definecolor{cellbackground}{HTML}{F7F7F7}
    
    % prompt
    \makeatletter
    \newcommand{\boxspacing}{\kern\kvtcb@left@rule\kern\kvtcb@boxsep}
    \makeatother
    \newcommand{\prompt}[4]{
        {\ttfamily\llap{{\color{#2}[#3]:\hspace{3pt}#4}}\vspace{-\baselineskip}}
    }
    

    
    % Prevent overflowing lines due to hard-to-break entities
    \sloppy 
    % Setup hyperref package
    \hypersetup{
      breaklinks=true,  % so long urls are correctly broken across lines
      colorlinks=true,
      urlcolor=urlcolor,
      linkcolor=linkcolor,
      citecolor=citecolor,
      }
    % Slightly bigger margins than the latex defaults
    
    \geometry{verbose,tmargin=1in,bmargin=1in,lmargin=1in,rmargin=1in}
    
    

\begin{document}
    
    \maketitle
    
    

    
    \hypertarget{ux4e60ux9898-6.7}{%
\section{习题 6.7}\label{ux4e60ux9898-6.7}}

\includegraphics{https://tva1.sinaimg.cn/large/008i3skNly1gre3npqkfbj31a209e0we.jpg}
\includegraphics{https://tva1.sinaimg.cn/large/008i3skNly1gre3o642hfj31j40a2gov.jpg}

    \begin{tcolorbox}[breakable, size=fbox, boxrule=1pt, pad at break*=1mm,colback=cellbackground, colframe=cellborder]
\prompt{In}{incolor}{1}{\boxspacing}
\begin{Verbatim}[commandchars=\\\{\}]
\PY{n}{data} \PY{o}{\PYZlt{}\PYZhy{}} \PY{n+nf}{read.csv}\PY{p}{(}\PY{l+s}{\PYZdq{}}\PY{l+s}{ex\PYZus{}6\PYZus{}7.csv\PYZdq{}}\PY{p}{,} \PY{n}{row.names}\PY{o}{=}\PY{l+m}{1}\PY{p}{)}\PY{p}{;} \PY{n}{data}
\end{Verbatim}
\end{tcolorbox}

    A data.frame: 16 × 6
\begin{tabular}{r|llllll}
  & x1 & x2 & x3 & x4 & x5 & x6\\
  & <dbl> & <dbl> & <dbl> & <dbl> & <dbl> & <dbl>\\
\hline
	1985 & 128.1 & 100.0 & 134.2 & 100.0 & 166.8 & 111.1\\
	1986 & 135.8 & 106.5 & 143.6 & 106.1 & 177.5 & 114.7\\
	1987 & 145.7 & 114.3 & 156.2 & 112.7 & 198.8 & 120.2\\
	1988 & 172.7 & 135.8 & 188.5 & 132.4 & 244.5 & 138.5\\
	1989 & 203.4 & 160.2 & 219.2 & 157.9 & 281.2 & 164.4\\
	1990 & 207.7 & 165.2 & 222.0 & 165.1 & 273.9 & 172.0\\
	1991 & 213.7 & 170.8 & 233.3 & 168.9 & 268.4 & 177.2\\
	1992 & 225.2 & 181.7 & 253.4 & 176.8 & 277.5 & 182.7\\
	1993 & 254.9 & 208.4 & 294.2 & 201.0 & 314.7 & 204.3\\
	1994 & 310.2 & 258.6 & 367.8 & 248.0 & 440.3 & 239.4\\
	1995 & 356.1 & 302.8 & 429.6 & 291.4 & 527.9 & 274.6\\
	1996 & 377.8 & 327.9 & 467.4 & 314.4 & 550.1 & 291.6\\
	1997 & 380.8 & 337.1 & 481.9 & 322.3 & 525.3 & 294.8\\
	1998 & 370.9 & 334.4 & 479.0 & 319.1 & 483.3 & 288.3\\
	1999 & 359.8 & 329.7 & 472.8 & 314.3 & 424.3 & 280.5\\
	2000 & 354.4 & 331.0 & 476.6 & 314.0 & 409.0 & 277.1\\
\end{tabular}


    
    \begin{tcolorbox}[breakable, size=fbox, boxrule=1pt, pad at break*=1mm,colback=cellbackground, colframe=cellborder]
\prompt{In}{incolor}{2}{\boxspacing}
\begin{Verbatim}[commandchars=\\\{\}]
\PY{n}{d} \PY{o}{\PYZlt{}\PYZhy{}} \PY{n+nf}{dist}\PY{p}{(}\PY{n}{data}\PY{p}{)}
\end{Verbatim}
\end{tcolorbox}

    \hypertarget{ux6700ux957fux8dddux79bbux6cd5}{%
\subsection{(1) 最长距离法}\label{ux6700ux957fux8dddux79bbux6cd5}}

    \begin{tcolorbox}[breakable, size=fbox, boxrule=1pt, pad at break*=1mm,colback=cellbackground, colframe=cellborder]
\prompt{In}{incolor}{3}{\boxspacing}
\begin{Verbatim}[commandchars=\\\{\}]
\PY{n}{h1} \PY{o}{\PYZlt{}\PYZhy{}} \PY{n+nf}{hclust}\PY{p}{(}\PY{n}{d}\PY{p}{,} \PY{n}{method}\PY{o}{=}\PY{l+s}{\PYZdq{}}\PY{l+s}{complete\PYZdq{}}\PY{p}{)}

\PY{n+nf}{plot}\PY{p}{(}\PY{n}{h1}\PY{p}{,} \PY{n}{hang}\PY{o}{=}\PY{l+m}{\PYZhy{}1}\PY{p}{,} \PY{n}{main}\PY{o}{=}\PY{l+s}{\PYZdq{}}\PY{l+s}{Longest Distance Cluster\PYZdq{}}\PY{p}{,} \PY{n}{xlab}\PY{o}{=}\PY{l+s}{\PYZdq{}}\PY{l+s}{Year\PYZdq{}}\PY{p}{)}    \PY{c+c1}{\PYZsh{} hang=\PYZhy{}1 表示线画到底}
\PY{n+nf}{rect.hclust}\PY{p}{(}\PY{n}{h1}\PY{p}{,} \PY{n}{k}\PY{o}{=}\PY{l+m}{3}\PY{p}{)}    \PY{c+c1}{\PYZsh{} 画出分为 3 类的矩形框}
\end{Verbatim}
\end{tcolorbox}

    \begin{center}
    \adjustimage{max size={0.9\linewidth}{0.9\paperheight}}{ex_6_7_files/ex_6_7_4_0.png}
    \end{center}
    { \hspace*{\fill} \\}
    
    \begin{tcolorbox}[breakable, size=fbox, boxrule=1pt, pad at break*=1mm,colback=cellbackground, colframe=cellborder]
\prompt{In}{incolor}{4}{\boxspacing}
\begin{Verbatim}[commandchars=\\\{\}]
\PY{c+c1}{\PYZsh{} HELPER:Years in the hclust order}
\PY{n+nf}{cat}\PY{p}{(}\PY{n}{h1}\PY{o}{\PYZdl{}}\PY{n}{labels}\PY{p}{[}\PY{n}{h1}\PY{o}{\PYZdl{}}\PY{n}{order}\PY{p}{]}\PY{p}{,} \PY{n}{sep}\PY{o}{=}\PY{l+s}{\PYZdq{}}\PY{l+s}{,\PYZdq{}}\PY{p}{)}
\end{Verbatim}
\end{tcolorbox}

    \begin{Verbatim}[commandchars=\\\{\}]
1994,1999,2000,1995,1998,1996,1997,1988,1987,1985,1986,1993,1989,1990,1991,1992
    \end{Verbatim}

    可以看出聚成三类的结果是:

\begin{itemize}
\tightlist
\item
  第一类:1994,1999,2000,1995,1998,1996,1997
\item
  第二类:1988,1987,1985,1986
\item
  第三类:1993,1989,1990,1991,1992
\end{itemize}

    \hypertarget{ux7c7bux5e73ux5747ux8dddux79bbux6cd5}{%
\subsection{(2)
类平均距离法}\label{ux7c7bux5e73ux5747ux8dddux79bbux6cd5}}

    \begin{tcolorbox}[breakable, size=fbox, boxrule=1pt, pad at break*=1mm,colback=cellbackground, colframe=cellborder]
\prompt{In}{incolor}{5}{\boxspacing}
\begin{Verbatim}[commandchars=\\\{\}]
\PY{n}{h2} \PY{o}{\PYZlt{}\PYZhy{}} \PY{n+nf}{hclust}\PY{p}{(}\PY{n}{d}\PY{p}{,} \PY{n}{method}\PY{o}{=}\PY{l+s}{\PYZdq{}}\PY{l+s}{average\PYZdq{}}\PY{p}{)}

\PY{n+nf}{plot}\PY{p}{(}\PY{n}{h2}\PY{p}{,} \PY{n}{hang}\PY{o}{=}\PY{l+m}{\PYZhy{}1}\PY{p}{,} \PY{n}{main}\PY{o}{=}\PY{l+s}{\PYZdq{}}\PY{l+s}{Classes Average Distance Cluster\PYZdq{}}\PY{p}{,} \PY{n}{xlab}\PY{o}{=}\PY{l+s}{\PYZdq{}}\PY{l+s}{Year\PYZdq{}}\PY{p}{)}
\PY{n+nf}{rect.hclust}\PY{p}{(}\PY{n}{h2}\PY{p}{,} \PY{n}{k}\PY{o}{=}\PY{l+m}{3}\PY{p}{)}    \PY{c+c1}{\PYZsh{} 画出分为 3 类的矩形框}
\end{Verbatim}
\end{tcolorbox}

    \begin{center}
    \adjustimage{max size={0.9\linewidth}{0.9\paperheight}}{ex_6_7_files/ex_6_7_8_0.png}
    \end{center}
    { \hspace*{\fill} \\}
    
    \begin{tcolorbox}[breakable, size=fbox, boxrule=1pt, pad at break*=1mm,colback=cellbackground, colframe=cellborder]
\prompt{In}{incolor}{6}{\boxspacing}
\begin{Verbatim}[commandchars=\\\{\}]
\PY{c+c1}{\PYZsh{} HELPER:Years in the hclust order}
\PY{n+nf}{cat}\PY{p}{(}\PY{n}{h2}\PY{o}{\PYZdl{}}\PY{n}{labels}\PY{p}{[}\PY{n}{h2}\PY{o}{\PYZdl{}}\PY{n}{order}\PY{p}{]}\PY{p}{,} \PY{n}{sep}\PY{o}{=}\PY{l+s}{\PYZdq{}}\PY{l+s}{,\PYZdq{}}\PY{p}{)}
\end{Verbatim}
\end{tcolorbox}

    \begin{Verbatim}[commandchars=\\\{\}]
1994,1999,2000,1995,1998,1996,1997,1987,1985,1986,1993,1988,1992,1991,1989,1990
    \end{Verbatim}

    聚成三类的结果是:

\begin{itemize}
\tightlist
\item
  第一类:1994,1999,2000,1995,1998,1996,1997
\item
  第二类:1987,1985,1986
\item
  第三类:1993,1988,1992,1991,1989,1990
\end{itemize}

    \hypertarget{ux6570ux636eux6807ux51c6ux5316}{%
\subsection{(3) 数据标准化}\label{ux6570ux636eux6807ux51c6ux5316}}

    \hypertarget{ux6570ux636eux6807ux51c6ux5316}{%
\subsubsection{数据标准化}\label{ux6570ux636eux6807ux51c6ux5316}}

    先用 scale 将数据标准化:

    \begin{tcolorbox}[breakable, size=fbox, boxrule=1pt, pad at break*=1mm,colback=cellbackground, colframe=cellborder]
\prompt{In}{incolor}{7}{\boxspacing}
\begin{Verbatim}[commandchars=\\\{\}]
\PY{n}{data.scale} \PY{o}{\PYZlt{}\PYZhy{}} \PY{n+nf}{scale}\PY{p}{(}\PY{n}{data}\PY{p}{)}\PY{p}{;} \PY{n}{data.scale}
\end{Verbatim}
\end{tcolorbox}

    A matrix: 16 × 6 of type dbl
\begin{tabular}{r|llllll}
  & x1 & x2 & x3 & x4 & x5 & x6\\
\hline
	1985 & -1.41939946 & -1.3334326 & -1.3213620 & -1.3417390 & -1.3788101 & -1.40310086\\
	1986 & -1.33797367 & -1.2628375 & -1.2521774 & -1.2707383 & -1.2972638 & -1.35108734\\
	1987 & -1.23328338 & -1.1781234 & -1.1594406 & -1.1939179 & -1.1349331 & -1.27162223\\
	1988 & -0.94776440 & -0.9446166 & -0.9217104 & -0.9646205 & -0.7866463 & -1.00722017\\
	1989 & -0.62311874 & -0.6796135 & -0.6957563 & -0.6678142 & -0.5069498 & -0.63301178\\
	1990 & -0.57764720 & -0.6253096 & -0.6751482 & -0.5840100 & -0.5625843 & -0.52320546\\
	1991 & -0.51419854 & -0.5644892 & -0.5919794 & -0.5397801 & -0.6045006 & -0.44807481\\
	1992 & -0.39258860 & -0.4461067 & -0.4440420 & -0.4478283 & -0.5351481 & -0.36860971\\
	1993 & -0.07851772 & -0.1561238 & -0.1437513 & -0.1661533 & -0.2516411 & -0.05652858\\
	1994 &  0.50626745 &  0.3890875 &  0.3979496 &  0.3809014 &  0.7055762 &  0.45060325\\
	1995 &  0.99164972 &  0.8691341 &  0.8528018 &  0.8860541 &  1.3731895 &  0.95917990\\
	1996 &  1.22112239 &  1.1417398 &  1.1310123 &  1.1537617 &  1.5423792 &  1.20479931\\
	1997 &  1.25284672 &  1.2416590 &  1.2377333 &  1.2457135 &  1.3533745 &  1.25103355\\
	1998 &  1.14815642 &  1.2123349 &  1.2163891 &  1.2084672 &  1.0332859 &  1.15712025\\
	1999 &  1.03077640 &  1.1612892 &  1.1707567 &  1.1525978 &  0.5836377 &  1.04442428\\
	2000 &  0.97367260 &  1.1754082 &  1.1987249 &  1.1491059 &  0.4670340 &  0.99530040\\
\end{tabular}


    
    重新计算标准化数据的距离:

    \begin{tcolorbox}[breakable, size=fbox, boxrule=1pt, pad at break*=1mm,colback=cellbackground, colframe=cellborder]
\prompt{In}{incolor}{8}{\boxspacing}
\begin{Verbatim}[commandchars=\\\{\}]
\PY{n}{ds} \PY{o}{\PYZlt{}\PYZhy{}} \PY{n+nf}{dist}\PY{p}{(}\PY{n}{data.scale}\PY{p}{)}
\end{Verbatim}
\end{tcolorbox}

    \hypertarget{ux6700ux957fux8dddux79bbux6cd5}{%
\subsubsection{最长距离法}\label{ux6700ux957fux8dddux79bbux6cd5}}

    \begin{tcolorbox}[breakable, size=fbox, boxrule=1pt, pad at break*=1mm,colback=cellbackground, colframe=cellborder]
\prompt{In}{incolor}{9}{\boxspacing}
\begin{Verbatim}[commandchars=\\\{\}]
\PY{n}{h3} \PY{o}{\PYZlt{}\PYZhy{}} \PY{n+nf}{hclust}\PY{p}{(}\PY{n}{ds}\PY{p}{,} \PY{n}{method}\PY{o}{=}\PY{l+s}{\PYZdq{}}\PY{l+s}{complete\PYZdq{}}\PY{p}{)}

\PY{n+nf}{plot}\PY{p}{(}\PY{n}{h3}\PY{p}{,} \PY{n}{hang}\PY{o}{=}\PY{l+m}{\PYZhy{}1}\PY{p}{,} \PY{n}{main}\PY{o}{=}\PY{l+s}{\PYZdq{}}\PY{l+s}{Longest Distance Cluster \PYZhy{} scaled data\PYZdq{}}\PY{p}{,} \PY{n}{xlab}\PY{o}{=}\PY{l+s}{\PYZdq{}}\PY{l+s}{Year\PYZdq{}}\PY{p}{)}
\PY{n+nf}{rect.hclust}\PY{p}{(}\PY{n}{h3}\PY{p}{,} \PY{n}{k}\PY{o}{=}\PY{l+m}{3}\PY{p}{)}
\end{Verbatim}
\end{tcolorbox}

    \begin{center}
    \adjustimage{max size={0.9\linewidth}{0.9\paperheight}}{ex_6_7_files/ex_6_7_18_0.png}
    \end{center}
    { \hspace*{\fill} \\}
    
    \begin{tcolorbox}[breakable, size=fbox, boxrule=1pt, pad at break*=1mm,colback=cellbackground, colframe=cellborder]
\prompt{In}{incolor}{10}{\boxspacing}
\begin{Verbatim}[commandchars=\\\{\}]
\PY{c+c1}{\PYZsh{} HELPER:Years in the hclust order}
\PY{n+nf}{cat}\PY{p}{(}\PY{n}{h3}\PY{o}{\PYZdl{}}\PY{n}{labels}\PY{p}{[}\PY{n}{h3}\PY{o}{\PYZdl{}}\PY{n}{order}\PY{p}{]}\PY{p}{,} \PY{n}{sep}\PY{o}{=}\PY{l+s}{\PYZdq{}}\PY{l+s}{,\PYZdq{}}\PY{p}{)}
\end{Verbatim}
\end{tcolorbox}

    \begin{Verbatim}[commandchars=\\\{\}]
1994,1999,2000,1995,1998,1996,1997,1988,1987,1985,1986,1993,1992,1989,1990,1991
    \end{Verbatim}

    聚成三类的结果:

\begin{itemize}
\tightlist
\item
  第一类:1994,1999,2000,1995,1998,1996,1997
\item
  第二类:1988,1987,1985,1986
\item
  第三类:1993,1992,1989,1990,1991
\end{itemize}

    \hypertarget{ux7c7bux5e73ux5747ux8dddux79bbux6cd5}{%
\subsubsection{类平均距离法}\label{ux7c7bux5e73ux5747ux8dddux79bbux6cd5}}

    \begin{tcolorbox}[breakable, size=fbox, boxrule=1pt, pad at break*=1mm,colback=cellbackground, colframe=cellborder]
\prompt{In}{incolor}{11}{\boxspacing}
\begin{Verbatim}[commandchars=\\\{\}]
\PY{n}{h4} \PY{o}{\PYZlt{}\PYZhy{}} \PY{n+nf}{hclust}\PY{p}{(}\PY{n}{ds}\PY{p}{,} \PY{n}{method}\PY{o}{=}\PY{l+s}{\PYZdq{}}\PY{l+s}{average\PYZdq{}}\PY{p}{)}

\PY{n+nf}{plot}\PY{p}{(}\PY{n}{h4}\PY{p}{,} \PY{n}{hang}\PY{o}{=}\PY{l+m}{\PYZhy{}1}\PY{p}{,} \PY{n}{main}\PY{o}{=}\PY{l+s}{\PYZdq{}}\PY{l+s}{Classes Average Distance Cluster \PYZhy{} scaled data\PYZdq{}}\PY{p}{,} \PY{n}{xlab}\PY{o}{=}\PY{l+s}{\PYZdq{}}\PY{l+s}{Year\PYZdq{}}\PY{p}{)}
\PY{n+nf}{rect.hclust}\PY{p}{(}\PY{n}{h4}\PY{p}{,} \PY{n}{k}\PY{o}{=}\PY{l+m}{3}\PY{p}{)}
\end{Verbatim}
\end{tcolorbox}

    \begin{center}
    \adjustimage{max size={0.9\linewidth}{0.9\paperheight}}{ex_6_7_files/ex_6_7_22_0.png}
    \end{center}
    { \hspace*{\fill} \\}
    
    \begin{tcolorbox}[breakable, size=fbox, boxrule=1pt, pad at break*=1mm,colback=cellbackground, colframe=cellborder]
\prompt{In}{incolor}{12}{\boxspacing}
\begin{Verbatim}[commandchars=\\\{\}]
\PY{c+c1}{\PYZsh{} HELPER:Years in the hclust order}
\PY{n+nf}{cat}\PY{p}{(}\PY{n}{h4}\PY{o}{\PYZdl{}}\PY{n}{labels}\PY{p}{[}\PY{n}{h4}\PY{o}{\PYZdl{}}\PY{n}{order}\PY{p}{]}\PY{p}{,} \PY{n}{sep}\PY{o}{=}\PY{l+s}{\PYZdq{}}\PY{l+s}{,\PYZdq{}}\PY{p}{)}
\end{Verbatim}
\end{tcolorbox}

    \begin{Verbatim}[commandchars=\\\{\}]
1994,1999,2000,1995,1998,1996,1997,1988,1987,1985,1986,1993,1992,1989,1990,1991
    \end{Verbatim}

    聚成三类的结果:

\begin{itemize}
\tightlist
\item
  第一类:1994,1999,2000,1995,1998,1996,1997
\item
  第二类:1988,1987,1985,1986
\item
  第三类:1993,1992,1989,1990,1991
\end{itemize}

    可以看到,在数据标准化之前不同聚类方法得到的结果不仅尽相同,而且在标准化前后聚类结果也是不要一样的,但是在数据标准化之后,两种不同的聚类方法得到了相同的结果。

    ref: https://www.jianshu.com/p/50cb85285af0


    % Add a bibliography block to the postdoc
    
    
    
\end{document}
