\documentclass[11pt]{article}

    \usepackage[breakable]{tcolorbox}
    \usepackage{parskip} % Stop auto-indenting (to mimic markdown behaviour)
    
    \usepackage{iftex}
    \ifPDFTeX
    	\usepackage[T1]{fontenc}
    	\usepackage{mathpazo}
    \else
    	\usepackage{fontspec}
    \fi

    % Basic figure setup, for now with no caption control since it's done
    % automatically by Pandoc (which extracts ![](path) syntax from Markdown).
    \usepackage{graphicx}
    % Maintain compatibility with old templates. Remove in nbconvert 6.0
    \let\Oldincludegraphics\includegraphics
    % Ensure that by default, figures have no caption (until we provide a
    % proper Figure object with a Caption API and a way to capture that
    % in the conversion process - todo).
    \usepackage{caption}
    \DeclareCaptionFormat{nocaption}{}
    \captionsetup{format=nocaption,aboveskip=0pt,belowskip=0pt}

    \usepackage{float}
    \floatplacement{figure}{H} % forces figures to be placed at the correct location
    \usepackage{xcolor} % Allow colors to be defined
    \usepackage{enumerate} % Needed for markdown enumerations to work
    \usepackage{geometry} % Used to adjust the document margins
    \usepackage{amsmath} % Equations
    \usepackage{amssymb} % Equations
    \usepackage{textcomp} % defines textquotesingle
    % Hack from http://tex.stackexchange.com/a/47451/13684:
    \AtBeginDocument{%
        \def\PYZsq{\textquotesingle}% Upright quotes in Pygmentized code
    }
    \usepackage{upquote} % Upright quotes for verbatim code
    \usepackage{eurosym} % defines \euro
    \usepackage[mathletters]{ucs} % Extended unicode (utf-8) support
    \usepackage{fancyvrb} % verbatim replacement that allows latex
    \usepackage{grffile} % extends the file name processing of package graphics 
                         % to support a larger range
    \makeatletter % fix for old versions of grffile with XeLaTeX
    \@ifpackagelater{grffile}{2019/11/01}
    {
      % Do nothing on new versions
    }
    {
      \def\Gread@@xetex#1{%
        \IfFileExists{"\Gin@base".bb}%
        {\Gread@eps{\Gin@base.bb}}%
        {\Gread@@xetex@aux#1}%
      }
    }
    \makeatother
    \usepackage[Export]{adjustbox} % Used to constrain images to a maximum size
    \adjustboxset{max size={0.9\linewidth}{0.9\paperheight}}

    % The hyperref package gives us a pdf with properly built
    % internal navigation ('pdf bookmarks' for the table of contents,
    % internal cross-reference links, web links for URLs, etc.)
    \usepackage{hyperref}
    % The default LaTeX title has an obnoxious amount of whitespace. By default,
    % titling removes some of it. It also provides customization options.
    \usepackage{titling}
    \usepackage{longtable} % longtable support required by pandoc >1.10
    \usepackage{booktabs}  % table support for pandoc > 1.12.2
    \usepackage[inline]{enumitem} % IRkernel/repr support (it uses the enumerate* environment)
    \usepackage[normalem]{ulem} % ulem is needed to support strikethroughs (\sout)
                                % normalem makes italics be italics, not underlines
    \usepackage{mathrsfs}
    

    
    % Colors for the hyperref package
    \definecolor{urlcolor}{rgb}{0,.145,.698}
    \definecolor{linkcolor}{rgb}{.71,0.21,0.01}
    \definecolor{citecolor}{rgb}{.12,.54,.11}

    % ANSI colors
    \definecolor{ansi-black}{HTML}{3E424D}
    \definecolor{ansi-black-intense}{HTML}{282C36}
    \definecolor{ansi-red}{HTML}{E75C58}
    \definecolor{ansi-red-intense}{HTML}{B22B31}
    \definecolor{ansi-green}{HTML}{00A250}
    \definecolor{ansi-green-intense}{HTML}{007427}
    \definecolor{ansi-yellow}{HTML}{DDB62B}
    \definecolor{ansi-yellow-intense}{HTML}{B27D12}
    \definecolor{ansi-blue}{HTML}{208FFB}
    \definecolor{ansi-blue-intense}{HTML}{0065CA}
    \definecolor{ansi-magenta}{HTML}{D160C4}
    \definecolor{ansi-magenta-intense}{HTML}{A03196}
    \definecolor{ansi-cyan}{HTML}{60C6C8}
    \definecolor{ansi-cyan-intense}{HTML}{258F8F}
    \definecolor{ansi-white}{HTML}{C5C1B4}
    \definecolor{ansi-white-intense}{HTML}{A1A6B2}
    \definecolor{ansi-default-inverse-fg}{HTML}{FFFFFF}
    \definecolor{ansi-default-inverse-bg}{HTML}{000000}

    % common color for the border for error outputs.
    \definecolor{outerrorbackground}{HTML}{FFDFDF}

    % commands and environments needed by pandoc snippets
    % extracted from the output of `pandoc -s`
    \providecommand{\tightlist}{%
      \setlength{\itemsep}{0pt}\setlength{\parskip}{0pt}}
    \DefineVerbatimEnvironment{Highlighting}{Verbatim}{commandchars=\\\{\}}
    % Add ',fontsize=\small' for more characters per line
    \newenvironment{Shaded}{}{}
    \newcommand{\KeywordTok}[1]{\textcolor[rgb]{0.00,0.44,0.13}{\textbf{{#1}}}}
    \newcommand{\DataTypeTok}[1]{\textcolor[rgb]{0.56,0.13,0.00}{{#1}}}
    \newcommand{\DecValTok}[1]{\textcolor[rgb]{0.25,0.63,0.44}{{#1}}}
    \newcommand{\BaseNTok}[1]{\textcolor[rgb]{0.25,0.63,0.44}{{#1}}}
    \newcommand{\FloatTok}[1]{\textcolor[rgb]{0.25,0.63,0.44}{{#1}}}
    \newcommand{\CharTok}[1]{\textcolor[rgb]{0.25,0.44,0.63}{{#1}}}
    \newcommand{\StringTok}[1]{\textcolor[rgb]{0.25,0.44,0.63}{{#1}}}
    \newcommand{\CommentTok}[1]{\textcolor[rgb]{0.38,0.63,0.69}{\textit{{#1}}}}
    \newcommand{\OtherTok}[1]{\textcolor[rgb]{0.00,0.44,0.13}{{#1}}}
    \newcommand{\AlertTok}[1]{\textcolor[rgb]{1.00,0.00,0.00}{\textbf{{#1}}}}
    \newcommand{\FunctionTok}[1]{\textcolor[rgb]{0.02,0.16,0.49}{{#1}}}
    \newcommand{\RegionMarkerTok}[1]{{#1}}
    \newcommand{\ErrorTok}[1]{\textcolor[rgb]{1.00,0.00,0.00}{\textbf{{#1}}}}
    \newcommand{\NormalTok}[1]{{#1}}
    
    % Additional commands for more recent versions of Pandoc
    \newcommand{\ConstantTok}[1]{\textcolor[rgb]{0.53,0.00,0.00}{{#1}}}
    \newcommand{\SpecialCharTok}[1]{\textcolor[rgb]{0.25,0.44,0.63}{{#1}}}
    \newcommand{\VerbatimStringTok}[1]{\textcolor[rgb]{0.25,0.44,0.63}{{#1}}}
    \newcommand{\SpecialStringTok}[1]{\textcolor[rgb]{0.73,0.40,0.53}{{#1}}}
    \newcommand{\ImportTok}[1]{{#1}}
    \newcommand{\DocumentationTok}[1]{\textcolor[rgb]{0.73,0.13,0.13}{\textit{{#1}}}}
    \newcommand{\AnnotationTok}[1]{\textcolor[rgb]{0.38,0.63,0.69}{\textbf{\textit{{#1}}}}}
    \newcommand{\CommentVarTok}[1]{\textcolor[rgb]{0.38,0.63,0.69}{\textbf{\textit{{#1}}}}}
    \newcommand{\VariableTok}[1]{\textcolor[rgb]{0.10,0.09,0.49}{{#1}}}
    \newcommand{\ControlFlowTok}[1]{\textcolor[rgb]{0.00,0.44,0.13}{\textbf{{#1}}}}
    \newcommand{\OperatorTok}[1]{\textcolor[rgb]{0.40,0.40,0.40}{{#1}}}
    \newcommand{\BuiltInTok}[1]{{#1}}
    \newcommand{\ExtensionTok}[1]{{#1}}
    \newcommand{\PreprocessorTok}[1]{\textcolor[rgb]{0.74,0.48,0.00}{{#1}}}
    \newcommand{\AttributeTok}[1]{\textcolor[rgb]{0.49,0.56,0.16}{{#1}}}
    \newcommand{\InformationTok}[1]{\textcolor[rgb]{0.38,0.63,0.69}{\textbf{\textit{{#1}}}}}
    \newcommand{\WarningTok}[1]{\textcolor[rgb]{0.38,0.63,0.69}{\textbf{\textit{{#1}}}}}
    
    
    % Define a nice break command that doesn't care if a line doesn't already
    % exist.
    \def\br{\hspace*{\fill} \\* }
    % Math Jax compatibility definitions
    \def\gt{>}
    \def\lt{<}
    \let\Oldtex\TeX
    \let\Oldlatex\LaTeX
    \renewcommand{\TeX}{\textrm{\Oldtex}}
    \renewcommand{\LaTeX}{\textrm{\Oldlatex}}
    % Document parameters
    % Document title
    \title{ex\_1\_4}
    
    
    
    
    
% Pygments definitions
\makeatletter
\def\PY@reset{\let\PY@it=\relax \let\PY@bf=\relax%
    \let\PY@ul=\relax \let\PY@tc=\relax%
    \let\PY@bc=\relax \let\PY@ff=\relax}
\def\PY@tok#1{\csname PY@tok@#1\endcsname}
\def\PY@toks#1+{\ifx\relax#1\empty\else%
    \PY@tok{#1}\expandafter\PY@toks\fi}
\def\PY@do#1{\PY@bc{\PY@tc{\PY@ul{%
    \PY@it{\PY@bf{\PY@ff{#1}}}}}}}
\def\PY#1#2{\PY@reset\PY@toks#1+\relax+\PY@do{#2}}

\@namedef{PY@tok@w}{\def\PY@tc##1{\textcolor[rgb]{0.73,0.73,0.73}{##1}}}
\@namedef{PY@tok@c}{\let\PY@it=\textit\def\PY@tc##1{\textcolor[rgb]{0.25,0.50,0.50}{##1}}}
\@namedef{PY@tok@cp}{\def\PY@tc##1{\textcolor[rgb]{0.74,0.48,0.00}{##1}}}
\@namedef{PY@tok@k}{\let\PY@bf=\textbf\def\PY@tc##1{\textcolor[rgb]{0.00,0.50,0.00}{##1}}}
\@namedef{PY@tok@kp}{\def\PY@tc##1{\textcolor[rgb]{0.00,0.50,0.00}{##1}}}
\@namedef{PY@tok@kt}{\def\PY@tc##1{\textcolor[rgb]{0.69,0.00,0.25}{##1}}}
\@namedef{PY@tok@o}{\def\PY@tc##1{\textcolor[rgb]{0.40,0.40,0.40}{##1}}}
\@namedef{PY@tok@ow}{\let\PY@bf=\textbf\def\PY@tc##1{\textcolor[rgb]{0.67,0.13,1.00}{##1}}}
\@namedef{PY@tok@nb}{\def\PY@tc##1{\textcolor[rgb]{0.00,0.50,0.00}{##1}}}
\@namedef{PY@tok@nf}{\def\PY@tc##1{\textcolor[rgb]{0.00,0.00,1.00}{##1}}}
\@namedef{PY@tok@nc}{\let\PY@bf=\textbf\def\PY@tc##1{\textcolor[rgb]{0.00,0.00,1.00}{##1}}}
\@namedef{PY@tok@nn}{\let\PY@bf=\textbf\def\PY@tc##1{\textcolor[rgb]{0.00,0.00,1.00}{##1}}}
\@namedef{PY@tok@ne}{\let\PY@bf=\textbf\def\PY@tc##1{\textcolor[rgb]{0.82,0.25,0.23}{##1}}}
\@namedef{PY@tok@nv}{\def\PY@tc##1{\textcolor[rgb]{0.10,0.09,0.49}{##1}}}
\@namedef{PY@tok@no}{\def\PY@tc##1{\textcolor[rgb]{0.53,0.00,0.00}{##1}}}
\@namedef{PY@tok@nl}{\def\PY@tc##1{\textcolor[rgb]{0.63,0.63,0.00}{##1}}}
\@namedef{PY@tok@ni}{\let\PY@bf=\textbf\def\PY@tc##1{\textcolor[rgb]{0.60,0.60,0.60}{##1}}}
\@namedef{PY@tok@na}{\def\PY@tc##1{\textcolor[rgb]{0.49,0.56,0.16}{##1}}}
\@namedef{PY@tok@nt}{\let\PY@bf=\textbf\def\PY@tc##1{\textcolor[rgb]{0.00,0.50,0.00}{##1}}}
\@namedef{PY@tok@nd}{\def\PY@tc##1{\textcolor[rgb]{0.67,0.13,1.00}{##1}}}
\@namedef{PY@tok@s}{\def\PY@tc##1{\textcolor[rgb]{0.73,0.13,0.13}{##1}}}
\@namedef{PY@tok@sd}{\let\PY@it=\textit\def\PY@tc##1{\textcolor[rgb]{0.73,0.13,0.13}{##1}}}
\@namedef{PY@tok@si}{\let\PY@bf=\textbf\def\PY@tc##1{\textcolor[rgb]{0.73,0.40,0.53}{##1}}}
\@namedef{PY@tok@se}{\let\PY@bf=\textbf\def\PY@tc##1{\textcolor[rgb]{0.73,0.40,0.13}{##1}}}
\@namedef{PY@tok@sr}{\def\PY@tc##1{\textcolor[rgb]{0.73,0.40,0.53}{##1}}}
\@namedef{PY@tok@ss}{\def\PY@tc##1{\textcolor[rgb]{0.10,0.09,0.49}{##1}}}
\@namedef{PY@tok@sx}{\def\PY@tc##1{\textcolor[rgb]{0.00,0.50,0.00}{##1}}}
\@namedef{PY@tok@m}{\def\PY@tc##1{\textcolor[rgb]{0.40,0.40,0.40}{##1}}}
\@namedef{PY@tok@gh}{\let\PY@bf=\textbf\def\PY@tc##1{\textcolor[rgb]{0.00,0.00,0.50}{##1}}}
\@namedef{PY@tok@gu}{\let\PY@bf=\textbf\def\PY@tc##1{\textcolor[rgb]{0.50,0.00,0.50}{##1}}}
\@namedef{PY@tok@gd}{\def\PY@tc##1{\textcolor[rgb]{0.63,0.00,0.00}{##1}}}
\@namedef{PY@tok@gi}{\def\PY@tc##1{\textcolor[rgb]{0.00,0.63,0.00}{##1}}}
\@namedef{PY@tok@gr}{\def\PY@tc##1{\textcolor[rgb]{1.00,0.00,0.00}{##1}}}
\@namedef{PY@tok@ge}{\let\PY@it=\textit}
\@namedef{PY@tok@gs}{\let\PY@bf=\textbf}
\@namedef{PY@tok@gp}{\let\PY@bf=\textbf\def\PY@tc##1{\textcolor[rgb]{0.00,0.00,0.50}{##1}}}
\@namedef{PY@tok@go}{\def\PY@tc##1{\textcolor[rgb]{0.53,0.53,0.53}{##1}}}
\@namedef{PY@tok@gt}{\def\PY@tc##1{\textcolor[rgb]{0.00,0.27,0.87}{##1}}}
\@namedef{PY@tok@err}{\def\PY@bc##1{{\setlength{\fboxsep}{-\fboxrule}\fcolorbox[rgb]{1.00,0.00,0.00}{1,1,1}{\strut ##1}}}}
\@namedef{PY@tok@kc}{\let\PY@bf=\textbf\def\PY@tc##1{\textcolor[rgb]{0.00,0.50,0.00}{##1}}}
\@namedef{PY@tok@kd}{\let\PY@bf=\textbf\def\PY@tc##1{\textcolor[rgb]{0.00,0.50,0.00}{##1}}}
\@namedef{PY@tok@kn}{\let\PY@bf=\textbf\def\PY@tc##1{\textcolor[rgb]{0.00,0.50,0.00}{##1}}}
\@namedef{PY@tok@kr}{\let\PY@bf=\textbf\def\PY@tc##1{\textcolor[rgb]{0.00,0.50,0.00}{##1}}}
\@namedef{PY@tok@bp}{\def\PY@tc##1{\textcolor[rgb]{0.00,0.50,0.00}{##1}}}
\@namedef{PY@tok@fm}{\def\PY@tc##1{\textcolor[rgb]{0.00,0.00,1.00}{##1}}}
\@namedef{PY@tok@vc}{\def\PY@tc##1{\textcolor[rgb]{0.10,0.09,0.49}{##1}}}
\@namedef{PY@tok@vg}{\def\PY@tc##1{\textcolor[rgb]{0.10,0.09,0.49}{##1}}}
\@namedef{PY@tok@vi}{\def\PY@tc##1{\textcolor[rgb]{0.10,0.09,0.49}{##1}}}
\@namedef{PY@tok@vm}{\def\PY@tc##1{\textcolor[rgb]{0.10,0.09,0.49}{##1}}}
\@namedef{PY@tok@sa}{\def\PY@tc##1{\textcolor[rgb]{0.73,0.13,0.13}{##1}}}
\@namedef{PY@tok@sb}{\def\PY@tc##1{\textcolor[rgb]{0.73,0.13,0.13}{##1}}}
\@namedef{PY@tok@sc}{\def\PY@tc##1{\textcolor[rgb]{0.73,0.13,0.13}{##1}}}
\@namedef{PY@tok@dl}{\def\PY@tc##1{\textcolor[rgb]{0.73,0.13,0.13}{##1}}}
\@namedef{PY@tok@s2}{\def\PY@tc##1{\textcolor[rgb]{0.73,0.13,0.13}{##1}}}
\@namedef{PY@tok@sh}{\def\PY@tc##1{\textcolor[rgb]{0.73,0.13,0.13}{##1}}}
\@namedef{PY@tok@s1}{\def\PY@tc##1{\textcolor[rgb]{0.73,0.13,0.13}{##1}}}
\@namedef{PY@tok@mb}{\def\PY@tc##1{\textcolor[rgb]{0.40,0.40,0.40}{##1}}}
\@namedef{PY@tok@mf}{\def\PY@tc##1{\textcolor[rgb]{0.40,0.40,0.40}{##1}}}
\@namedef{PY@tok@mh}{\def\PY@tc##1{\textcolor[rgb]{0.40,0.40,0.40}{##1}}}
\@namedef{PY@tok@mi}{\def\PY@tc##1{\textcolor[rgb]{0.40,0.40,0.40}{##1}}}
\@namedef{PY@tok@il}{\def\PY@tc##1{\textcolor[rgb]{0.40,0.40,0.40}{##1}}}
\@namedef{PY@tok@mo}{\def\PY@tc##1{\textcolor[rgb]{0.40,0.40,0.40}{##1}}}
\@namedef{PY@tok@ch}{\let\PY@it=\textit\def\PY@tc##1{\textcolor[rgb]{0.25,0.50,0.50}{##1}}}
\@namedef{PY@tok@cm}{\let\PY@it=\textit\def\PY@tc##1{\textcolor[rgb]{0.25,0.50,0.50}{##1}}}
\@namedef{PY@tok@cpf}{\let\PY@it=\textit\def\PY@tc##1{\textcolor[rgb]{0.25,0.50,0.50}{##1}}}
\@namedef{PY@tok@c1}{\let\PY@it=\textit\def\PY@tc##1{\textcolor[rgb]{0.25,0.50,0.50}{##1}}}
\@namedef{PY@tok@cs}{\let\PY@it=\textit\def\PY@tc##1{\textcolor[rgb]{0.25,0.50,0.50}{##1}}}

\def\PYZbs{\char`\\}
\def\PYZus{\char`\_}
\def\PYZob{\char`\{}
\def\PYZcb{\char`\}}
\def\PYZca{\char`\^}
\def\PYZam{\char`\&}
\def\PYZlt{\char`\<}
\def\PYZgt{\char`\>}
\def\PYZsh{\char`\#}
\def\PYZpc{\char`\%}
\def\PYZdl{\char`\$}
\def\PYZhy{\char`\-}
\def\PYZsq{\char`\'}
\def\PYZdq{\char`\"}
\def\PYZti{\char`\~}
% for compatibility with earlier versions
\def\PYZat{@}
\def\PYZlb{[}
\def\PYZrb{]}
\makeatother


    % For linebreaks inside Verbatim environment from package fancyvrb. 
    \makeatletter
        \newbox\Wrappedcontinuationbox 
        \newbox\Wrappedvisiblespacebox 
        \newcommand*\Wrappedvisiblespace {\textcolor{red}{\textvisiblespace}} 
        \newcommand*\Wrappedcontinuationsymbol {\textcolor{red}{\llap{\tiny$\m@th\hookrightarrow$}}} 
        \newcommand*\Wrappedcontinuationindent {3ex } 
        \newcommand*\Wrappedafterbreak {\kern\Wrappedcontinuationindent\copy\Wrappedcontinuationbox} 
        % Take advantage of the already applied Pygments mark-up to insert 
        % potential linebreaks for TeX processing. 
        %        {, <, #, %, $, ' and ": go to next line. 
        %        _, }, ^, &, >, - and ~: stay at end of broken line. 
        % Use of \textquotesingle for straight quote. 
        \newcommand*\Wrappedbreaksatspecials {% 
            \def\PYGZus{\discretionary{\char`\_}{\Wrappedafterbreak}{\char`\_}}% 
            \def\PYGZob{\discretionary{}{\Wrappedafterbreak\char`\{}{\char`\{}}% 
            \def\PYGZcb{\discretionary{\char`\}}{\Wrappedafterbreak}{\char`\}}}% 
            \def\PYGZca{\discretionary{\char`\^}{\Wrappedafterbreak}{\char`\^}}% 
            \def\PYGZam{\discretionary{\char`\&}{\Wrappedafterbreak}{\char`\&}}% 
            \def\PYGZlt{\discretionary{}{\Wrappedafterbreak\char`\<}{\char`\<}}% 
            \def\PYGZgt{\discretionary{\char`\>}{\Wrappedafterbreak}{\char`\>}}% 
            \def\PYGZsh{\discretionary{}{\Wrappedafterbreak\char`\#}{\char`\#}}% 
            \def\PYGZpc{\discretionary{}{\Wrappedafterbreak\char`\%}{\char`\%}}% 
            \def\PYGZdl{\discretionary{}{\Wrappedafterbreak\char`\$}{\char`\$}}% 
            \def\PYGZhy{\discretionary{\char`\-}{\Wrappedafterbreak}{\char`\-}}% 
            \def\PYGZsq{\discretionary{}{\Wrappedafterbreak\textquotesingle}{\textquotesingle}}% 
            \def\PYGZdq{\discretionary{}{\Wrappedafterbreak\char`\"}{\char`\"}}% 
            \def\PYGZti{\discretionary{\char`\~}{\Wrappedafterbreak}{\char`\~}}% 
        } 
        % Some characters . , ; ? ! / are not pygmentized. 
        % This macro makes them "active" and they will insert potential linebreaks 
        \newcommand*\Wrappedbreaksatpunct {% 
            \lccode`\~`\.\lowercase{\def~}{\discretionary{\hbox{\char`\.}}{\Wrappedafterbreak}{\hbox{\char`\.}}}% 
            \lccode`\~`\,\lowercase{\def~}{\discretionary{\hbox{\char`\,}}{\Wrappedafterbreak}{\hbox{\char`\,}}}% 
            \lccode`\~`\;\lowercase{\def~}{\discretionary{\hbox{\char`\;}}{\Wrappedafterbreak}{\hbox{\char`\;}}}% 
            \lccode`\~`\:\lowercase{\def~}{\discretionary{\hbox{\char`\:}}{\Wrappedafterbreak}{\hbox{\char`\:}}}% 
            \lccode`\~`\?\lowercase{\def~}{\discretionary{\hbox{\char`\?}}{\Wrappedafterbreak}{\hbox{\char`\?}}}% 
            \lccode`\~`\!\lowercase{\def~}{\discretionary{\hbox{\char`\!}}{\Wrappedafterbreak}{\hbox{\char`\!}}}% 
            \lccode`\~`\/\lowercase{\def~}{\discretionary{\hbox{\char`\/}}{\Wrappedafterbreak}{\hbox{\char`\/}}}% 
            \catcode`\.\active
            \catcode`\,\active 
            \catcode`\;\active
            \catcode`\:\active
            \catcode`\?\active
            \catcode`\!\active
            \catcode`\/\active 
            \lccode`\~`\~ 	
        }
    \makeatother

    \let\OriginalVerbatim=\Verbatim
    \makeatletter
    \renewcommand{\Verbatim}[1][1]{%
        %\parskip\z@skip
        \sbox\Wrappedcontinuationbox {\Wrappedcontinuationsymbol}%
        \sbox\Wrappedvisiblespacebox {\FV@SetupFont\Wrappedvisiblespace}%
        \def\FancyVerbFormatLine ##1{\hsize\linewidth
            \vtop{\raggedright\hyphenpenalty\z@\exhyphenpenalty\z@
                \doublehyphendemerits\z@\finalhyphendemerits\z@
                \strut ##1\strut}%
        }%
        % If the linebreak is at a space, the latter will be displayed as visible
        % space at end of first line, and a continuation symbol starts next line.
        % Stretch/shrink are however usually zero for typewriter font.
        \def\FV@Space {%
            \nobreak\hskip\z@ plus\fontdimen3\font minus\fontdimen4\font
            \discretionary{\copy\Wrappedvisiblespacebox}{\Wrappedafterbreak}
            {\kern\fontdimen2\font}%
        }%
        
        % Allow breaks at special characters using \PYG... macros.
        \Wrappedbreaksatspecials
        % Breaks at punctuation characters . , ; ? ! and / need catcode=\active 	
        \OriginalVerbatim[#1,codes*=\Wrappedbreaksatpunct]%
    }
    \makeatother

    % Exact colors from NB
    \definecolor{incolor}{HTML}{303F9F}
    \definecolor{outcolor}{HTML}{D84315}
    \definecolor{cellborder}{HTML}{CFCFCF}
    \definecolor{cellbackground}{HTML}{F7F7F7}
    
    % prompt
    \makeatletter
    \newcommand{\boxspacing}{\kern\kvtcb@left@rule\kern\kvtcb@boxsep}
    \makeatother
    \newcommand{\prompt}[4]{
        {\ttfamily\llap{{\color{#2}[#3]:\hspace{3pt}#4}}\vspace{-\baselineskip}}
    }
    

    
    % Prevent overflowing lines due to hard-to-break entities
    \sloppy 
    % Setup hyperref package
    \hypersetup{
      breaklinks=true,  % so long urls are correctly broken across lines
      colorlinks=true,
      urlcolor=urlcolor,
      linkcolor=linkcolor,
      citecolor=citecolor,
      }
    % Slightly bigger margins than the latex defaults
    
    \geometry{verbose,tmargin=1in,bmargin=1in,lmargin=1in,rmargin=1in}
    
    

\begin{document}
    
    \maketitle
    
    

    
    \hypertarget{ux4e60ux9898-1.4}{%
\subsection{习题 1.4}\label{ux4e60ux9898-1.4}}

    \begin{tcolorbox}[breakable, size=fbox, boxrule=1pt, pad at break*=1mm,colback=cellbackground, colframe=cellborder]
\prompt{In}{incolor}{1}{\boxspacing}
\begin{Verbatim}[commandchars=\\\{\}]
\PY{n}{data} \PY{o}{\PYZlt{}\PYZhy{}} \PY{n+nf}{read.csv}\PY{p}{(}\PY{l+s}{\PYZdq{}}\PY{l+s}{./ex\PYZus{}1\PYZus{}4.csv\PYZdq{}}\PY{p}{)}
\PY{n+nf}{cat}\PY{p}{(}\PY{n+nf}{names}\PY{p}{(}\PY{n}{data}\PY{p}{)}\PY{p}{)}
\end{Verbatim}
\end{tcolorbox}

    \begin{Verbatim}[commandchars=\\\{\}]
X.序号 省市区 X11月 X1.11月
    \end{Verbatim}

    设「11月」为 \(X_1\), 「1~11月」为 \(X_2\):

    \begin{tcolorbox}[breakable, size=fbox, boxrule=1pt, pad at break*=1mm,colback=cellbackground, colframe=cellborder]
\prompt{In}{incolor}{2}{\boxspacing}
\begin{Verbatim}[commandchars=\\\{\}]
\PY{n}{data} \PY{o}{\PYZlt{}\PYZhy{}} \PY{n}{data}\PY{p}{[}\PY{l+m}{\PYZhy{}1}\PY{p}{]}  \PY{c+c1}{\PYZsh{} remove \PYZdq{}序号\PYZdq{} col}
\PY{n+nf}{names}\PY{p}{(}\PY{n}{data}\PY{p}{)} \PY{o}{\PYZlt{}\PYZhy{}} \PY{n+nf}{c}\PY{p}{(}\PY{l+s}{\PYZdq{}}\PY{l+s}{Province\PYZdq{}}\PY{p}{,} \PY{l+s}{\PYZdq{}}\PY{l+s}{X1\PYZdq{}}\PY{p}{,} \PY{l+s}{\PYZdq{}}\PY{l+s}{X2\PYZdq{}}\PY{p}{)}
\PY{n}{data}
\end{Verbatim}
\end{tcolorbox}

    A data.frame: 31 × 3
\begin{tabular}{lll}
 Province & X1 & X2\\
 <chr> & <dbl> & <dbl>\\
\hline
	 北京       & 35.22 &  499.80\\
	 天津       & 10.41 &  161.37\\
	 河北       & 17.22 &  273.29\\
	 山西       & 10.70 &  134.79\\
	 内蒙古 & 10.29 &   90.92\\
	 辽宁       & 18.66 &  348.99\\
	 吉林       &  4.41 &  106.89\\
	 黑龙江 &  6.24 &  196.44\\
	 上海       & 49.72 &  656.95\\
	 江苏       & 47.70 &  580.70\\
	 浙江       & 36.55 &  518.10\\
	 安徽       & 14.85 &  179.41\\
	 福建       & 19.46 &  250.16\\
	 江西       & 10.93 &  122.06\\
	 山东       & 40.26 &  552.74\\
	 河南       & 19.82 &  268.20\\
	 湖北       & 19.49 &  221.43\\
	 湖南       & 16.01 &  197.68\\
	 广东       & 99.32 & 1080.26\\
	 广西       & 14.77 &  160.60\\
	 海南       &  3.96 &   39.51\\
	 重庆       & 10.49 &  111.76\\
	 四川       & 21.71 &  250.09\\
	 贵州       & 13.06 &   95.87\\
	 云南       & 20.34 &  183.62\\
	 西藏       &  0.77 &    6.08\\
	 陕西       & 11.38 &  133.50\\
	 甘肃       &  3.66 &   64.86\\
	 青海       &  1.21 &   18.30\\
	 宁夏       &  2.31 &   23.81\\
	 新疆       &  3.24 &  103.81\\
\end{tabular}


    
    \begin{tcolorbox}[breakable, size=fbox, boxrule=1pt, pad at break*=1mm,colback=cellbackground, colframe=cellborder]
\prompt{In}{incolor}{3}{\boxspacing}
\begin{Verbatim}[commandchars=\\\{\}]
\PY{n+nf}{attach}\PY{p}{(}\PY{n}{data}\PY{p}{)}
\end{Verbatim}
\end{tcolorbox}

    \hypertarget{ux5747ux503cux65b9ux5deeux6807ux51c6ux5deeux53d8ux5f02ux7cfbux6570ux504fux5ea6ux5cf0ux5ea6}{%
\subsubsection{(1)
均值、方差、标准差、变异系数、偏度、峰度}\label{ux5747ux503cux65b9ux5deeux6807ux51c6ux5deeux53d8ux5f02ux7cfbux6570ux504fux5ea6ux5cf0ux5ea6}}

    调用库完成:

    \begin{tcolorbox}[breakable, size=fbox, boxrule=1pt, pad at break*=1mm,colback=cellbackground, colframe=cellborder]
\prompt{In}{incolor}{4}{\boxspacing}
\begin{Verbatim}[commandchars=\\\{\}]
\PY{n+nf}{library}\PY{p}{(}\PY{n}{psych}\PY{p}{)}
\PY{n+nf}{describe}\PY{p}{(}\PY{n}{data}\PY{p}{[}\PY{l+m}{\PYZhy{}1}\PY{p}{]}\PY{p}{,} \PY{n}{type}\PY{o}{=}\PY{l+m}{2}\PY{p}{)}
\end{Verbatim}
\end{tcolorbox}

    A psych: 2 × 13
\begin{tabular}{r|lllllllllllll}
  & vars & n & mean & sd & median & trimmed & mad & min & max & range & skew & kurtosis & se\\
  & <int> & <dbl> & <dbl> & <dbl> & <dbl> & <dbl> & <dbl> & <dbl> & <dbl> & <dbl> & <dbl> & <dbl> & <dbl>\\
\hline
	X1 & 1 & 31 &  19.16645 &  19.79977 &  14.77 &  15.7252 &   8.258082 & 0.77 &   99.32 &   98.55 & 2.515352 & 8.266989 &  3.556143\\
	X2 & 2 & 31 & 246.19323 & 232.97210 & 179.41 & 210.6356 & 123.856404 & 6.08 & 1080.26 & 1074.18 & 1.915957 & 4.385233 & 41.843024\\
\end{tabular}


    
    手动实现:

    \begin{tcolorbox}[breakable, size=fbox, boxrule=1pt, pad at break*=1mm,colback=cellbackground, colframe=cellborder]
\prompt{In}{incolor}{5}{\boxspacing}
\begin{Verbatim}[commandchars=\\\{\}]
\PY{n}{describes} \PY{o}{\PYZlt{}\PYZhy{}} \PY{n+nf}{function}\PY{p}{(}\PY{n}{df}\PY{p}{)} \PY{p}{\PYZob{}}
    \PY{n}{cv} \PY{o}{\PYZlt{}\PYZhy{}} \PY{n+nf}{function}\PY{p}{(}\PY{n}{x}\PY{p}{)} \PY{n+nf}{sd}\PY{p}{(}\PY{n}{x}\PY{p}{)}\PY{o}{/}\PY{n+nf}{mean}\PY{p}{(}\PY{n}{x}\PY{p}{)}  \PY{c+c1}{\PYZsh{} 变异系数}

    \PY{n}{g1} \PY{o}{\PYZlt{}\PYZhy{}} \PY{n+nf}{function}\PY{p}{(}\PY{n}{x}\PY{p}{)} \PY{p}{\PYZob{}}  \PY{c+c1}{\PYZsh{} 偏度}
        \PY{n}{n} \PY{o}{\PYZlt{}\PYZhy{}} \PY{n+nf}{length}\PY{p}{(}\PY{n}{x}\PY{p}{)}
        \PY{n}{A} \PY{o}{\PYZlt{}\PYZhy{}} \PY{n}{n} \PY{o}{/} \PY{p}{(}\PY{p}{(}\PY{n}{n}\PY{l+m}{\PYZhy{}1}\PY{p}{)} \PY{o}{*} \PY{p}{(}\PY{n}{n}\PY{l+m}{\PYZhy{}2}\PY{p}{)}\PY{p}{)}
        \PY{n}{B} \PY{o}{\PYZlt{}\PYZhy{}} \PY{l+m}{1} \PY{o}{/} \PY{n+nf}{sd}\PY{p}{(}\PY{n}{x}\PY{p}{)}\PY{o}{\PYZca{}}\PY{l+m}{3}
        \PY{n}{S} \PY{o}{\PYZlt{}\PYZhy{}} \PY{n+nf}{sum}\PY{p}{(}\PY{p}{(}\PY{n}{x} \PY{o}{\PYZhy{}} \PY{n+nf}{mean}\PY{p}{(}\PY{n}{x}\PY{p}{)}\PY{p}{)}\PY{o}{\PYZca{}}\PY{l+m}{3}\PY{p}{)}
        \PY{n}{A} \PY{o}{*} \PY{n}{B} \PY{o}{*} \PY{n}{S}
    \PY{p}{\PYZcb{}}

    \PY{n}{g2} \PY{o}{\PYZlt{}\PYZhy{}} \PY{n+nf}{function}\PY{p}{(}\PY{n}{x}\PY{p}{)} \PY{p}{\PYZob{}}  \PY{c+c1}{\PYZsh{} 峰度}
        \PY{n}{n} \PY{o}{\PYZlt{}\PYZhy{}} \PY{n+nf}{length}\PY{p}{(}\PY{n}{x}\PY{p}{)}
        \PY{n}{A} \PY{o}{\PYZlt{}\PYZhy{}} \PY{p}{(}\PY{n}{n} \PY{o}{*} \PY{p}{(}\PY{n}{n}\PY{l+m}{+1}\PY{p}{)}\PY{p}{)} \PY{o}{/} \PY{p}{(}\PY{p}{(}\PY{n}{n}\PY{l+m}{\PYZhy{}1}\PY{p}{)} \PY{o}{*} \PY{p}{(}\PY{n}{n}\PY{l+m}{\PYZhy{}2}\PY{p}{)} \PY{o}{*} \PY{p}{(}\PY{n}{n}\PY{l+m}{\PYZhy{}3}\PY{p}{)}\PY{p}{)}
        \PY{n}{B} \PY{o}{\PYZlt{}\PYZhy{}} \PY{l+m}{1} \PY{o}{/} \PY{n+nf}{sd}\PY{p}{(}\PY{n}{x}\PY{p}{)}\PY{o}{\PYZca{}}\PY{l+m}{4}
        \PY{n}{S} \PY{o}{\PYZlt{}\PYZhy{}} \PY{n+nf}{sum}\PY{p}{(}\PY{p}{(}\PY{n}{x} \PY{o}{\PYZhy{}} \PY{n+nf}{mean}\PY{p}{(}\PY{n}{x}\PY{p}{)}\PY{p}{)}\PY{o}{\PYZca{}}\PY{l+m}{4}\PY{p}{)}
        \PY{n}{C} \PY{o}{\PYZlt{}\PYZhy{}} \PY{p}{(}\PY{l+m}{3} \PY{o}{*} \PY{p}{(}\PY{n}{n}\PY{l+m}{\PYZhy{}1}\PY{p}{)}\PY{o}{\PYZca{}}\PY{l+m}{2}\PY{p}{)} \PY{o}{/} \PY{p}{(}\PY{p}{(}\PY{n}{n}\PY{l+m}{\PYZhy{}2}\PY{p}{)} \PY{o}{*} \PY{p}{(}\PY{n}{n}\PY{l+m}{\PYZhy{}3}\PY{p}{)}\PY{p}{)}
        \PY{n}{A} \PY{o}{*} \PY{n}{B} \PY{o}{*} \PY{n}{S} \PY{o}{\PYZhy{}} \PY{n}{C}
    \PY{p}{\PYZcb{}}
    
    \PY{n}{itm} \PY{o}{\PYZlt{}\PYZhy{}} \PY{n+nf}{matrix}\PY{p}{(}\PY{n+nf}{c}\PY{p}{(}\PY{l+s}{\PYZdq{}}\PY{l+s}{均值\PYZdq{}}\PY{p}{,} \PY{l+s}{\PYZdq{}}\PY{l+s}{方差\PYZdq{}}\PY{p}{,} \PY{l+s}{\PYZdq{}}\PY{l+s}{标准差\PYZdq{}}\PY{p}{,} \PY{l+s}{\PYZdq{}}\PY{l+s}{变异系数\PYZdq{}}\PY{p}{,} \PY{l+s}{\PYZdq{}}\PY{l+s}{偏度\PYZdq{}}\PY{p}{,} \PY{l+s}{\PYZdq{}}\PY{l+s}{峰度\PYZdq{}}\PY{p}{)}\PY{p}{,} \PY{l+m}{6}\PY{p}{,} \PY{l+m}{1}\PY{p}{)}
    \PY{n}{res} \PY{o}{\PYZlt{}\PYZhy{}} \PY{n+nf}{apply}\PY{p}{(}\PY{n}{df}\PY{p}{,} \PY{l+m}{2}\PY{p}{,} 
                 \PY{n+nf}{function}\PY{p}{(}\PY{n}{x}\PY{p}{)} \PY{n+nf}{c}\PY{p}{(}\PY{n+nf}{mean}\PY{p}{(}\PY{n}{x}\PY{p}{)}\PY{p}{,} \PY{n+nf}{var}\PY{p}{(}\PY{n}{x}\PY{p}{)}\PY{p}{,} \PY{n+nf}{sd}\PY{p}{(}\PY{n}{x}\PY{p}{)}\PY{p}{,} \PY{n+nf}{cv}\PY{p}{(}\PY{n}{x}\PY{p}{)}\PY{p}{,} \PY{n+nf}{g1}\PY{p}{(}\PY{n}{x}\PY{p}{)}\PY{p}{,} \PY{n+nf}{g2}\PY{p}{(}\PY{n}{x}\PY{p}{)}\PY{p}{)}\PY{p}{)}
    
    \PY{n+nf}{cbind}\PY{p}{(}\PY{n}{itm}\PY{p}{,} \PY{n}{res}\PY{p}{)}
\PY{p}{\PYZcb{}}
\end{Verbatim}
\end{tcolorbox}

    \begin{tcolorbox}[breakable, size=fbox, boxrule=1pt, pad at break*=1mm,colback=cellbackground, colframe=cellborder]
\prompt{In}{incolor}{6}{\boxspacing}
\begin{Verbatim}[commandchars=\\\{\}]
\PY{n+nf}{describes}\PY{p}{(}\PY{n}{data}\PY{p}{[}\PY{l+m}{\PYZhy{}1}\PY{p}{]}\PY{p}{)}
\end{Verbatim}
\end{tcolorbox}

    A matrix: 6 × 3 of type chr
\begin{tabular}{lll}
  & X1 & X2\\
\hline
	 均值             & 19.1664516129032 & 246.193225806452 \\
	 方差             & 392.030750322581 & 54275.9982492473 \\
	 标准差       & 19.7997664209096 & 232.972097576614 \\
	 变异系数 & 1.0330428824697  & 0.946297757842323\\
	 偏度             & 2.51535182567297 & 1.91595698889706 \\
	 峰度             & 8.26698939080861 & 4.38523271345801 \\
\end{tabular}


    
    \hypertarget{ux4e2dux4f4dux6570ux4e0aux4e0bux56dbux5206ux4f4dux6570ux56dbux5206ux4f4dux6781ux5dee}{%
\subsubsection{(2)
中位数、上下四分位数、四分位极差}\label{ux4e2dux4f4dux6570ux4e0aux4e0bux56dbux5206ux4f4dux6570ux56dbux5206ux4f4dux6781ux5dee}}

    五数:minimum, lower-hinge, median, upper-hinge, maximum

    \begin{tcolorbox}[breakable, size=fbox, boxrule=1pt, pad at break*=1mm,colback=cellbackground, colframe=cellborder]
\prompt{In}{incolor}{7}{\boxspacing}
\begin{Verbatim}[commandchars=\\\{\}]
\PY{n}{fn} \PY{o}{\PYZlt{}\PYZhy{}} \PY{n+nf}{apply}\PY{p}{(}\PY{n}{data}\PY{p}{[}\PY{l+m}{\PYZhy{}1}\PY{p}{]}\PY{p}{,} \PY{l+m}{2}\PY{p}{,} \PY{n}{fivenum}\PY{p}{)}
\PY{n}{fn}
\end{Verbatim}
\end{tcolorbox}

    A matrix: 5 × 2 of type dbl
\begin{tabular}{ll}
 X1 & X2\\
\hline
	  0.770 &    6.080\\
	  8.265 &  105.350\\
	 14.770 &  179.410\\
	 20.080 &  270.745\\
	 99.320 & 1080.260\\
\end{tabular}


    
    \begin{tcolorbox}[breakable, size=fbox, boxrule=1pt, pad at break*=1mm,colback=cellbackground, colframe=cellborder]
\prompt{In}{incolor}{8}{\boxspacing}
\begin{Verbatim}[commandchars=\\\{\}]
\PY{c+c1}{\PYZsh{} 四分位极差}
\PY{n}{R1} \PY{o}{\PYZlt{}\PYZhy{}} \PY{n+nf}{function}\PY{p}{(}\PY{n}{Q3}\PY{p}{,} \PY{n}{Q1}\PY{p}{)} \PY{n}{Q3} \PY{o}{\PYZhy{}} \PY{n}{Q1}

\PY{n+nf}{R1}\PY{p}{(}\PY{n}{Q3}\PY{o}{=}\PY{n}{fn}\PY{p}{[}\PY{l+m}{4}\PY{p}{,}\PY{p}{]}\PY{p}{,} \PY{n}{Q1}\PY{o}{=}\PY{n}{fn}\PY{p}{[}\PY{l+m}{2}\PY{p}{,}\PY{p}{]}\PY{p}{)}
\end{Verbatim}
\end{tcolorbox}

    \begin{description*}
\item[X1] 11.815
\item[X2] 165.395
\end{description*}


    
    \hypertarget{ux76f4ux65b9ux56fe}{%
\subsubsection{(3) 直方图}\label{ux76f4ux65b9ux56fe}}

    \begin{tcolorbox}[breakable, size=fbox, boxrule=1pt, pad at break*=1mm,colback=cellbackground, colframe=cellborder]
\prompt{In}{incolor}{9}{\boxspacing}
\begin{Verbatim}[commandchars=\\\{\}]
\PY{n}{histogram} \PY{o}{\PYZlt{}\PYZhy{}} \PY{n+nf}{function}\PY{p}{(}\PY{n}{x}\PY{p}{,} \PY{n}{xname}\PY{o}{=}\PY{l+s}{\PYZdq{}}\PY{l+s}{x\PYZdq{}}\PY{p}{)} \PY{p}{\PYZob{}}
    \PY{n+nf}{hist}\PY{p}{(}\PY{n}{x}\PY{p}{,} \PY{n}{prob}\PY{o}{=}\PY{k+kc}{TRUE}\PY{p}{,} \PY{n}{main}\PY{o}{=}\PY{n+nf}{paste}\PY{p}{(}\PY{l+s}{\PYZdq{}}\PY{l+s}{Histogram of\PYZdq{}} \PY{p}{,} \PY{n}{xname}\PY{p}{)}\PY{p}{)}
    \PY{n+nf}{lines}\PY{p}{(}\PY{n+nf}{density}\PY{p}{(}\PY{n}{x}\PY{p}{)}\PY{p}{)}
    \PY{n+nf}{rug}\PY{p}{(}\PY{n}{x}\PY{p}{)} \PY{c+c1}{\PYZsh{} show the actual data points}
\PY{p}{\PYZcb{}}
\end{Verbatim}
\end{tcolorbox}

    \begin{tcolorbox}[breakable, size=fbox, boxrule=1pt, pad at break*=1mm,colback=cellbackground, colframe=cellborder]
\prompt{In}{incolor}{10}{\boxspacing}
\begin{Verbatim}[commandchars=\\\{\}]
\PY{n+nf}{histogram}\PY{p}{(}\PY{n}{X1}\PY{p}{,} \PY{l+s}{\PYZdq{}}\PY{l+s}{X1\PYZdq{}}\PY{p}{)}
\PY{n+nf}{histogram}\PY{p}{(}\PY{n}{X2}\PY{p}{,} \PY{l+s}{\PYZdq{}}\PY{l+s}{X2\PYZdq{}}\PY{p}{)}
\end{Verbatim}
\end{tcolorbox}

    \begin{center}
    \adjustimage{max size={0.9\linewidth}{0.9\paperheight}}{output_17_0.png}
    \end{center}
    { \hspace*{\fill} \\}
    
    \begin{center}
    \adjustimage{max size={0.9\linewidth}{0.9\paperheight}}{output_17_1.png}
    \end{center}
    { \hspace*{\fill} \\}
    
    \hypertarget{ux7ecfux9a8cux5206ux5e03ux51fdux6570ux56fe}{%
\subsubsection{(4)
经验分布函数图}\label{ux7ecfux9a8cux5206ux5e03ux51fdux6570ux56fe}}

    \begin{tcolorbox}[breakable, size=fbox, boxrule=1pt, pad at break*=1mm,colback=cellbackground, colframe=cellborder]
\prompt{In}{incolor}{11}{\boxspacing}
\begin{Verbatim}[commandchars=\\\{\}]
\PY{n}{plot\PYZus{}ecdf} \PY{o}{\PYZlt{}\PYZhy{}} \PY{n+nf}{function}\PY{p}{(}\PY{n}{x}\PY{p}{,} \PY{n}{xname}\PY{o}{=}\PY{l+s}{\PYZdq{}}\PY{l+s}{x\PYZdq{}}\PY{p}{)} \PY{p}{\PYZob{}}
    \PY{n+nf}{plot}\PY{p}{(}\PY{n+nf}{ecdf}\PY{p}{(}\PY{n}{x}\PY{p}{)}\PY{p}{,} \PY{n}{do.points}\PY{o}{=}\PY{k+kc}{FALSE}\PY{p}{,} \PY{n}{verticals}\PY{o}{=}\PY{k+kc}{TRUE}\PY{p}{,} \PY{n}{main}\PY{o}{=}\PY{n+nf}{paste}\PY{p}{(}\PY{l+s}{\PYZdq{}}\PY{l+s}{ecdf(\PYZdq{}} \PY{p}{,} \PY{n}{xname}\PY{p}{,} \PY{l+s}{\PYZdq{}}\PY{l+s}{)\PYZdq{}}\PY{p}{)}\PY{p}{)}
    
    \PY{n}{xs} \PY{o}{\PYZlt{}\PYZhy{}} \PY{n+nf}{seq}\PY{p}{(}\PY{n+nf}{min}\PY{p}{(}\PY{n}{x}\PY{p}{)}\PY{p}{,} \PY{n+nf}{max}\PY{p}{(}\PY{n}{x}\PY{p}{)}\PY{p}{,} \PY{l+m}{1}\PY{o}{/}\PY{n+nf}{sqrt}\PY{p}{(}\PY{n+nf}{length}\PY{p}{(}\PY{n}{x}\PY{p}{)}\PY{p}{)}\PY{p}{)}
    \PY{n+nf}{lines}\PY{p}{(}\PY{n}{xs}\PY{p}{,} \PY{n+nf}{pnorm}\PY{p}{(}\PY{n}{xs}\PY{p}{,} \PY{n}{mean}\PY{o}{=}\PY{n+nf}{mean}\PY{p}{(}\PY{n}{x}\PY{p}{)}\PY{p}{,} \PY{n}{sd}\PY{o}{=}\PY{n+nf}{sd}\PY{p}{(}\PY{n}{x}\PY{p}{)}\PY{p}{)}\PY{p}{,} \PY{n}{lty}\PY{o}{=}\PY{l+m}{3}\PY{p}{,} \PY{n}{col}\PY{o}{=}\PY{l+s}{\PYZdq{}}\PY{l+s}{red\PYZdq{}}\PY{p}{)}
\PY{p}{\PYZcb{}}
\end{Verbatim}
\end{tcolorbox}

    \begin{tcolorbox}[breakable, size=fbox, boxrule=1pt, pad at break*=1mm,colback=cellbackground, colframe=cellborder]
\prompt{In}{incolor}{12}{\boxspacing}
\begin{Verbatim}[commandchars=\\\{\}]
\PY{n+nf}{plot\PYZus{}ecdf}\PY{p}{(}\PY{n}{X1}\PY{p}{,} \PY{l+s}{\PYZdq{}}\PY{l+s}{X1\PYZdq{}}\PY{p}{)}
\end{Verbatim}
\end{tcolorbox}

    \begin{center}
    \adjustimage{max size={0.9\linewidth}{0.9\paperheight}}{output_20_0.png}
    \end{center}
    { \hspace*{\fill} \\}
    
    \begin{tcolorbox}[breakable, size=fbox, boxrule=1pt, pad at break*=1mm,colback=cellbackground, colframe=cellborder]
\prompt{In}{incolor}{13}{\boxspacing}
\begin{Verbatim}[commandchars=\\\{\}]
\PY{n+nf}{plot\PYZus{}ecdf}\PY{p}{(}\PY{n}{X2}\PY{p}{,} \PY{l+s}{\PYZdq{}}\PY{l+s}{X2\PYZdq{}}\PY{p}{)}
\end{Verbatim}
\end{tcolorbox}

    \begin{center}
    \adjustimage{max size={0.9\linewidth}{0.9\paperheight}}{output_21_0.png}
    \end{center}
    { \hspace*{\fill} \\}
    
    正态QQ图:

    \begin{tcolorbox}[breakable, size=fbox, boxrule=1pt, pad at break*=1mm,colback=cellbackground, colframe=cellborder]
\prompt{In}{incolor}{14}{\boxspacing}
\begin{Verbatim}[commandchars=\\\{\}]
\PY{n+nf}{qqnorm}\PY{p}{(}\PY{n}{X1}\PY{p}{)}
\end{Verbatim}
\end{tcolorbox}

    \begin{center}
    \adjustimage{max size={0.9\linewidth}{0.9\paperheight}}{output_23_0.png}
    \end{center}
    { \hspace*{\fill} \\}
    
    \begin{tcolorbox}[breakable, size=fbox, boxrule=1pt, pad at break*=1mm,colback=cellbackground, colframe=cellborder]
\prompt{In}{incolor}{15}{\boxspacing}
\begin{Verbatim}[commandchars=\\\{\}]
\PY{n+nf}{qqnorm}\PY{p}{(}\PY{n}{X2}\PY{p}{)}
\end{Verbatim}
\end{tcolorbox}

    \begin{center}
    \adjustimage{max size={0.9\linewidth}{0.9\paperheight}}{output_24_0.png}
    \end{center}
    { \hspace*{\fill} \\}
    
    \hypertarget{x_1x_2-ux7684-pearson-ux76f8ux5173ux7cfbux6570ux4e0e-spearman-ux76f8ux5173ux7cfbux6570}{%
\subsubsection{\texorpdfstring{(5) \(X_1\)、\(X_2\) 的 Pearson
相关系数与 Spearman
相关系数}{(5) X\_1、X\_2 的 Pearson 相关系数与 Spearman 相关系数}}\label{x_1x_2-ux7684-pearson-ux76f8ux5173ux7cfbux6570ux4e0e-spearman-ux76f8ux5173ux7cfbux6570}}

    \begin{tcolorbox}[breakable, size=fbox, boxrule=1pt, pad at break*=1mm,colback=cellbackground, colframe=cellborder]
\prompt{In}{incolor}{16}{\boxspacing}
\begin{Verbatim}[commandchars=\\\{\}]
\PY{n+nf}{cor.test}\PY{p}{(}\PY{n}{X1}\PY{p}{,} \PY{n}{X2}\PY{p}{,} \PY{n}{method}\PY{o}{=}\PY{l+s}{\PYZdq{}}\PY{l+s}{pearson\PYZdq{}}\PY{p}{)}
\end{Verbatim}
\end{tcolorbox}

    
    \begin{Verbatim}[commandchars=\\\{\}]

	Pearson's product-moment correlation

data:  X1 and X2
t = 24.265, df = 29, p-value < 2.2e-16
alternative hypothesis: true correlation is not equal to 0
95 percent confidence interval:
 0.9508176 0.9886055
sample estimates:
      cor 
0.9762474 

    \end{Verbatim}

    
    \begin{tcolorbox}[breakable, size=fbox, boxrule=1pt, pad at break*=1mm,colback=cellbackground, colframe=cellborder]
\prompt{In}{incolor}{17}{\boxspacing}
\begin{Verbatim}[commandchars=\\\{\}]
\PY{n+nf}{cor.test}\PY{p}{(}\PY{n}{X1}\PY{p}{,} \PY{n}{X2}\PY{p}{,} \PY{n}{method}\PY{o}{=}\PY{l+s}{\PYZdq{}}\PY{l+s}{spearman\PYZdq{}}\PY{p}{)}
\end{Verbatim}
\end{tcolorbox}

    
    \begin{Verbatim}[commandchars=\\\{\}]

	Spearman's rank correlation rho

data:  X1 and X2
S = 358, p-value = 9.781e-09
alternative hypothesis: true rho is not equal to 0
sample estimates:
      rho 
0.9278226 

    \end{Verbatim}

    
    \begin{tcolorbox}[breakable, size=fbox, boxrule=1pt, pad at break*=1mm,colback=cellbackground, colframe=cellborder]
\prompt{In}{incolor}{18}{\boxspacing}
\begin{Verbatim}[commandchars=\\\{\}]
\PY{n+nf}{detach}\PY{p}{(}\PY{n}{data}\PY{p}{)}
\end{Verbatim}
\end{tcolorbox}


    % Add a bibliography block to the postdoc
    
    
    
\end{document}
